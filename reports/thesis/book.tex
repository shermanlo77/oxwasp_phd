\documentclass[12pt, a4paper]{memoir}
%\documentclass[12pt, a4paper, oldfontcommands]{memoir}
%\documentclass[12pt, a4paper]{report}

\usepackage{amsmath}
\usepackage{amsthm}
\usepackage{amssymb}
\usepackage{amsbsy}
\usepackage{graphicx}
\usepackage[outdir=./]{epstopdf}
\usepackage[round]{natbib} 
\usepackage{url}
\usepackage{subcaption}
\usepackage{hyperref}
\usepackage{hyphenat}
\usepackage{pdfpages}

\let\footruleskip\undefined
\usepackage{fancyhdr}
\fancypagestyle{plain}{%
\fancyhf{}% clears all header and footer fields
\fancyhead[LE,RO]{\thepage}%
\renewcommand{\headrulewidth}{0pt}%
\renewcommand{\footrulewidth}{0pt}}

\DeclareMathOperator{\expfamily}{ExpFamily}
\DeclareMathOperator{\expectation}{\mathbb{E}}
\DeclareMathOperator{\variance}{\mathbb{V}ar}
\DeclareMathOperator{\cov}{\mathbb{C}ov}
\DeclareMathOperator{\corr}{\mathbb{C}orr}
\DeclareMathOperator{\bernoulli}{Bernoulli}
\DeclareMathOperator{\betaDist}{Beta}
\DeclareMathOperator{\dirichlet}{Dir}
\DeclareMathOperator{\bin}{Bin}
\DeclareMathOperator{\MN}{Multinomial}
\DeclareMathOperator{\prob}{\mathbb{P}}
\DeclareMathOperator{\trace}{Tr}
\DeclareMathOperator{\normal}{N}
\DeclareMathOperator{\gammaDist}{Gamma}
\DeclareMathOperator{\poisson}{Poisson}

\newcommand{\RSS}{\mathrm{RSS}}
\newcommand{\euler}{\mathrm{e}}
\newcommand{\diff}{\mathrm{d}}
\newcommand{\T}{^\textup{T}}
\newcommand{\dotdotdot}{_{\phantom{.}\cdots}}
\newcommand{\BIC}{\textup{BIC}}

\newcommand{\vect}[1]{\mathbf{#1}}
\newcommand{\vectGreek}[1]{\boldsymbol{#1}}
\newcommand{\matr}[1]{\mathsf{#1}}

\begin{document}

\includepdf{title.pdf}

%\frontmatter
\section*{Abstract}
X-ray CT can be used for defect detection and quality control of 3D printing. This report will outline statistical methods for comparing x-ray scans of a 3D printed sample with its computer model under the face of uncertainty. Shading correction methods and the mean and variance relationship were investigated. Lastly the compound Poisson was reviewed.

\newpage

\section*{Acknowledgements}
\begin{itemize}
	\item Supervisors: Julia Brettschneider and Tom Nichols
	\item Inside Out Team: Wilfrid Kendall, Audrey Kueh, Jay Warnett and Clair Barnes
	\item EPSRC Funding: EP/L016710/1
\end{itemize}

\newpage
\tableofcontents*

%\mainmatter

\chapter{Introduction}


\chapter{Literature Review}

3D printing, formally additive manufacturing \citep{gibson2010additive} \citep{wong2012review}, has recently become state of the art technology which manufactures objects of complicated shapes with very high precision. It was developed in the 1980's \citep{kodama1981automatic} but recently it has been commercialised and is used in medical science \citep{kang20163d} and engineering \citep{wong2012review}. Because of such a wide range of applications, there has been a need for quality control and this can be done by scanning the 3D printed sample using x-ray computed tomography.

\section{X-ray Imaging}

\subsection{X-Ray Production}
Photons in CT scanning are produced in an X-ray tube. In an X-ray tube, a cathode, consisting of a heated filament, fires projectile electrons through an electric potential to a target which forms the anode \citep{michael2001x}, as shown in Figure \ref{fig:x_ray_tube}. Most of the kinetic energy of the projectile electrons is converted into heat however some is converted into electromagnetic radiation. This depends on how the projectile electrons interact with the atoms in the anode \citep{cantatore2011introduction}.

\begin{figure}
\centering
\includegraphics[width=0.8\textwidth]{figures/x_ray_tube.png}
\caption{An X-ray tube produces photons by firing projectile electrons from a cathode to an anode. \emph{Source: G.~Michael (2001) \citep{michael2001x}}}
\label{fig:x_ray_tube}
\end{figure}

Bremsstrahlung radiation is the result of projectile electrons deaccelerating due to the electrostatic field produced by nucleus of the target. The kinetic energy of the projectile electrons is then converted to electromagnetic radiation to produce X-ray radiation. As a result, the photon energies in bremsstrahlung radiation is  a continuous spectrum and can range up to the maximum kinetic energy of the projectile electrons \citep{michael2001x}.

Characteristic radiation is due to projectile electrons colliding with electrons in the target atom and ionizing them. This produce vacancies in the electron shell and emits photons when the electrons in the target atom drops down back to the ground state. The energy of the emitted radiation is monoenergetic and depends on the binding energy of the target's atoms \citep{michael2001x}.

A typical energy spectrum of photons emitted from an X-ray tube is as shown in Figure \ref{fig:x_ray_spectrum}. The energy spectrum consist of both bremsstrahlung and characteristic radiation \citep{michael2001x}.

\begin{figure}
\centering
\includegraphics[width=0.8\textwidth]{figures/x_ray_spectrum.png}
\caption{A typical energy spectrum of photons emitted from an X-ray tube. The continuous spectrum is the result of Bremmsstrahlung radiation. The peaks are the result of characteristic radiation. \emph{Source: G.~Michael (2001) \citep{michael2001x}}}
\label{fig:x_ray_spectrum}
\end{figure}

The voltage and current can be varied in the X-ray tube to produce different energy spectrums and rate of photon production. This can vary the results produced when collecting CT data \citep{cantatore2011introduction}. Another important factor is the focal spot size because smaller spot sizes produce sharper edges. Larger spot sizes produce unsharp results and this is know as the penumbra effect, as shown in Figure \ref{fig:x_ray_penumbra}. However spot sizes too small can produce concentrated heat \citep{welkenhuyzen2009industrial} and can damage the X-ray tube.

\begin{figure}
\centering
\includegraphics[width=1\textwidth]{figures/x_ray_penumbra.png}
\caption{Larger focal spot sizes produces unsharp results. This is know as the penumbra effect. \emph{Source: F.~Welkenhuyzen et al.~(2009)\citep{welkenhuyzen2009industrial}}}
\label{fig:x_ray_penumbra}
\end{figure}

\subsection{Photon Interactions}
Photons emitted by the X-ray tube are projected onto the sample and interacts with it in a number of ways \citep{cantatore2011introduction}.

The sample can effectively absorb photons via the photoelectric effect or pair production \citep{cantatore2011introduction}. In the photoelectric effect, the photons transfers all its energy to a bounded electron and ejects it from the sample's atom \citep{millikan1916direct}. In pair production, the photons convert into electron-position pairs by interacting with the Coulomb field of the sample's atomic nucleus \citep{hubbell2006electron}. In addition to getting absorbed, photons can be scattered by the sample. This happens when photons collide inelastically with and transfers its energy to the sample's electrons. This process is known as Compton scattering \citep{compton1923quantum}.

Suppose a mono-energetic X-ray pencil beam attenuating through an object with varying attenuation coefficient in position $\mu=\mu(x)$. The beam starts at $x=0$ and is detected at $x=L$, then the attenuation (decrease in X-ray intensity from $I_0$ to $I_1$) is given as \citep{cantatore2011introduction}
\begin{equation}
I_1 = I_0\exp\left[\int_0^L-\mu(x)\diff x\right].
\label{eq:beerLaw}
\end{equation}
By comparing the intensity before and after attenuation, the integral of the attenuation coefficient along the path of the X-ray can be calculated. Because of the discrete nature of pixels, the integral is usually replaced by a sum \citep{michael2001x}. 

However it was shown that the attenuation coefficient does depend on the energy of the photons \citep{elbakri2002statistical}. Thus $\mu=\mu(x,E)$ should be made dependent on the energy of the photons \citep{cantatore2011introduction}. In general low energy photons are more likely to be absorbed than high energy photons, this increases the average energy of the attenuated photons and can be a source of error in CT scanning. This is referred to beam hardening and can cause inaccuracies in Equation \eqref{eq:beerLaw} \citep{michael2001x}. This can be reduced by placing a thin sheet of filter to absorb low energy photons \citep{welkenhuyzen2009industrial} or by correcting it in the data analysis stage \citep{michael2001x}.

\subsection{Detection and Reconstruction}
Most X-ray detectors are scintillator-photodiode detectors. The photons interact with the scintiallator material and produce visible light. The visible light is then detected by photodiodes and coverts it into electricical current \citep{michael2001x}. The detectors used in CT scanning are flat bed scanner which consist of an array of panels of photodiodes \citep{cantatore2011introduction}.

\section{X-ray Computed Tomography}
Computed tomography (CT) scanning is a 3D imaging technique. It does this by reconstructing the geometry of the sample through a series of 2D X-ray images of the sample. The sample rotates after each image taken \citep{cantatore2011introduction}.

Figure \ref{fig:x_ray_ct} shows a diagram on how CT scanning works. A 2D image is taken by projecting X-ray photons onto the stationary sample. The photons are then scattered or absobred by the sample.  Some of these photons are then detected by an X-ray detector on the other side of the sample, which produces an image. After an image has been taken, the object rotates and another image is taken. Finally after a number of images, a 3D reconstruction of the object can be estimated \citep{cantatore2011introduction}.

\begin{figure}
\centering
\includegraphics[width=0.7\textwidth]{figures/x_ray_ct.png}
\caption{X-ray computed tomography reconstructs the sample by projecting photons onto a rotating sample. The photons are then detected by the detector. \emph{Source: \url{http://www.phoenix-xray.com/}}}
\label{fig:x_ray_ct}
\end{figure}

CT scanning was invented by G. Hounsfield \citep{hounsfield1980computed} in the 1980's and it was mainly used for medical imaging. The setup for CT scanning is different when scanning patients because the detector and X-ray source rotates around the patient \citep{cantatore2011introduction}. Recently CT has been used industrially for non-destructive testing in manufacturing \citep{cantatore2011introduction}. One possible application would be inspecting 3D printed samples \citep{villarraga2015assessing}. Other uses in science include the investigation of batteries \citep{o2017investigating} and materials \citep{wang2017x} \citep{zhang2016x}.

Methods used to reconstruct the 3D surface of the sample are available such as filtered back-projection \citep{brooks1976principles} and the FDK algorithm \citep{feldkamp1984practical}. There exist software such as \emph{VGStudio MAX} \citep{reinhart2008industrial} which automatically aligns and scales the 3D reconstruction with the computed aided design and compares them.

Studies have been done to use x-ray CT to do defection on 3D printing \citep{kim2016inspection} \citep{villarraga2015assessing}. This was done by investigating the variation of the material thickness and the distribution of the volume of voids \citep{villarraga2015assessing}. Voids then can be classified as defects if the voids are larger than some volume threshold. This threshold controls the probability of the detection of defects \citep{amrhein2014characterization} \citep{gandossi2010probability}.

Full 3D x-ray CT scans can be slow and may not generalised well in production lines. However there is potential to bring in methods such as those used in airport luggage inspection to speed up to process \cite{warnett2016towards}.

There are many sources of error in CT scanning \citep{cantatore2011introduction} and this can cause problems when reconstructing the geometry of the sample. Sources of error include: dead pixels in the detector \citep{brettschneider2014spatial}, use of cone beams \citep{sun2016applications}, beam hardening and the orientation of the sample \citep{corcoran2016observations}.

The scale of the 2D images can be calibrated with the use of reference standards \citep{bartscher2007enhancement} \citep{lifton2013application}.

\section{Compound Poisson}


\chapter{Compound Poisson}
\chapter{Inference}

\bibliographystyle{apalike}
\bibliography{../bib}

\end{document}
