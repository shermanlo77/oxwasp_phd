\documentclass[12pt, a4paper]{memoir}
%\documentclass[12pt, a4paper, oldfontcommands]{memoir}
%\documentclass[12pt, a4paper]{report}

\usepackage{amsmath}
\usepackage{amsthm}
\usepackage{amssymb}
\usepackage{amsbsy}
\usepackage{graphicx}
\usepackage[outdir=./]{epstopdf}
\usepackage[round]{natbib} 
\usepackage{url}
\usepackage{subcaption}
\usepackage{hyperref}
\usepackage{hyphenat}
\usepackage{pdfpages}
\usepackage{MnSymbol} %udots

\setcounter{tocdepth}{2}
\setcounter{secnumdepth}{2}

\let\footruleskip\undefined
\usepackage{fancyhdr}
\fancypagestyle{plain}{%
\fancyhf{}% clears all header and footer fields
\fancyhead[LE,RO]{\thepage}%
\renewcommand{\headrulewidth}{0pt}%
\renewcommand{\footrulewidth}{0pt}}

\DeclareMathOperator{\expfamily}{ExpFamily}
\DeclareMathOperator{\expectation}{\mathbb{E}}
\DeclareMathOperator{\variance}{\mathbb{V}ar}
\DeclareMathOperator{\cov}{\mathbb{C}ov}
\DeclareMathOperator{\corr}{\mathbb{C}orr}
\DeclareMathOperator{\bernoulli}{Bernoulli}
\DeclareMathOperator{\betaDist}{Beta}
\DeclareMathOperator{\dirichlet}{Dir}
\DeclareMathOperator{\bin}{Bin}
\DeclareMathOperator{\MN}{Multinomial}
\DeclareMathOperator{\prob}{\mathbb{P}}
\DeclareMathOperator{\trace}{Tr}
\DeclareMathOperator{\normal}{N}
\DeclareMathOperator{\gammaDist}{Gamma}
\DeclareMathOperator{\poisson}{Poisson}
\DeclareMathOperator{\CPoisson}{CP\Gamma}
\DeclareMathOperator*{\argmax}{argmax}

\newcommand{\RSS}{\mathrm{RSS}}
\newcommand{\euler}{\mathrm{e}}
\newcommand{\diff}{\mathrm{d}}
\newcommand{\T}{^\textup{T}}
\newcommand{\dotdotdot}{_{\phantom{.}\cdots}}
\newcommand{\BIC}{\textup{BIC}}

\newcommand{\vect}[1]{\mathbf{#1}}
\newcommand{\vectGreek}[1]{\boldsymbol{#1}}
\newcommand{\matr}[1]{\mathsf{#1}}

\begin{document}\sloppy

\includepdf{title.pdf}

\frontmatter
\section*{Abstract}
X-ray computed tomography (CT) can be used for defect detection in 3D printing. The object is scanned at multiple angles to reconstruct the object in 3D space. The process can be time consuming. The aim of this project was to investigate if it is possible to conduct defect detection in projection space from a single scan to speed up the defect detection procedure. To do this, a 3D printed sample was manufactured with voids to see if they can be detected.

X-ray photons behave randomly. Hence to do defect detection pixel by pixel, uncertainty must be taken into account. A compound Poisson model was used to model the grey valyes in a pixel. It assumed that photon arrivals are a Poisson process with Gamma distributed energy. This resulted in a linear relationship between the mean and variance of the grey value, which was used for variance prediction and to quantify the uncertainty.

Software, called \texttt{aRTist}, was used to simulate the scan and it was compared with the x-ray acquisition under the face of uncertainty. However the software, and the information provided to it, was not perfect which led to model misspecification and incorrect inference. The empirical null filter was proposed. It adjust each pixel statistic so that inference was done by comparing each pixel with the majority of its local pixels.

\newpage

\section*{Acknowledgements}
\begin{itemize}
	\item Supervisors: Julia Brettschneider and Tom Nichols
	\item Inside Out Team: Wilfrid Kendall, Audrey Kueh, Jay Warnett and Clair Barnes
	\item EPSRC Funding: EP/L016710/1
\end{itemize}

\newpage
\tableofcontents*

\mainmatter

\chapter{Introduction}


\chapter{Literature Review}
One of the first methods of additive manufacturing (AM) is stereolithography \citep{kodama1981automatic, hull1986apparatus, 3d2019our} which involves the curing of a photosensitive resin using an ultraviolet laser. The technology has evolved and AM is capable of manufacturing objects with complicated internal and external geometries, some examples are shown in Figure \ref{fig:literature_3dprint}. However, there is a need for product inspection and in particular assessing the quality of the internal structures.

\begin{figure}
  \centering
  \includegraphics[width=0.9\textwidth]{../figures/literatureReview/literature_3dprint.png}
  \caption{Examples of additive manufactured parts: a) lattice structure, b) toy, c) chain, d) model of a facial implant, e) spanner, f) ratchet mechanism, g) toy, h) series of rotatable gears, i) lattice structure. Republished with permission of Springer New York, from \cite{gibson2010additive}; permission conveyed through Copyright
Clearance Center, Inc.}
  \label{fig:literature_3dprint}
\end{figure}

Imaging using x-rays \citep{rontgen1896on} has been used in the medical field. In x-ray computed tomography \citep{cormack1973reconstruction, hounsfield1973computerized, hounsfield1980computed}, the patient has an x-ray image taken at multiple angles. These x-ray images are used to reconstruct what was taken in 3D to make a diagnostic.

X-ray computed tomography (XCT) can be used as a non-destructive test for AM products. Various reviews on AM exist such as \cite{kruth1991material, kruth1998progress, pham1998comparison, gibson2010additive, wong2012review, ngo2018additive}. For XCT used in manufacturing, there are \cite{cantatore2011introduction, kruth2011computed, sun2012overview}. \cite{thompson2016x} reviewed the applications of XCT on AM.

In this chapter, AM is reviewed followed by XCT. The latest research for the use of XCT on AM is reviewed at the end of the chapter.

\section{Additive Manufacturing}

Loosely, AM involves solidifying material onto a moving platform so that the object is manufactured layer by layer. Typically, this is a slow and expensive method compared to destructive methods such as computer numerical control (CNC) machining for example. An advantage of AM is that the setup cost is low, in particular, destructive methods require planning and setting up various apparatus before the manufacturing stage \citep{gibson2010additive}. This makes it suitable to manufacture bespoke items which achieves the goals of AM's predecessor called rapid prototyping \citep{kruth1991material}.

Various AM technologies were invented during the advancement of AM. Because of this, there are various applications of AM, for example in medical and biomedical sciences \citep{kang20163d, kourra2018computed}, engineering \citep{cooper2015design}, food engineering \citep{godoi20163d} and art \citep{ornes2013mathematics, grossman2019bathsheba}.

\subsection{Additive Manufacturing Technologies}

The different AM technologies can be classified based on the apparatus, for example, liquid-based or powder-based, and/or on the method of manufacturing, for example, point by point or layer by layer \citep{kruth1991material}. The liquid-based AM technologies presented here are stereolithography \citep{kodama1981automatic, hull1986apparatus, 3d2019our} and fused deposition modelling \citep{crump1991fused, crump1992apparatus, stratasys2019what}. The following powder-based technologies are presented here: 3D printing \citep{sachs1990three}, selective laser sintering \citep{deckard1989method, dtm1990the, 3d2019our}, electron beam melting \citep{larsson2004arrangement, arcam2019history}, laser engineered net shaping \citep{atwood1998laser}. Illustrations of these technologies are shown in Figure \ref{fig:literature_technologies}.

\begin{figure}
  \centering
  \centerline{
    \begin{subfigure}[b]{0.59\textwidth}
      \includegraphics[width=\textwidth]{../figures/literatureReview/literature_technologies_stl.png}
      \caption{Stereolithography}
    \end{subfigure}
    \begin{subfigure}[b]{0.39\textwidth}
      \includegraphics[width=\textwidth]{../figures/literatureReview/literature_technologies_fdm.png}
      \caption{Fused deposition modelling}
    \end{subfigure}
  }
  \centerline{
    \begin{subfigure}[b]{0.7\textwidth}
      \includegraphics[width=\textwidth]{../figures/literatureReview/literature_technologies_3dp.png}
      \caption{3D printing}
    \end{subfigure}
  }
  \centerline{
    \begin{subfigure}[b]{0.7\textwidth}
      \includegraphics[width=\textwidth]{../figures/literatureReview/literature_technologies_laser.png}
      \caption{Selective laser sintering}
    \end{subfigure}
  }
  \caption{Diagrams of various AM technologies. Reprinted from \cite{wang20173d}\textsuperscript{\textcopyright}, with permission from Elsevier.}
  \label{fig:literature_technologies}
\end{figure}

Stereolithography is a liquid-based AM technology. It consists of a container containing a liquid photo-hardening monomer or polymer as well as a piston and platform which holds and moves the manufactured product up and down. A laser with a specific wavelength, typically \SIrange{300}{400}{\nano\metre} \citep{kodama1981automatic}, is emitted onto a point of the surface of the liquid and solidifies. The laser is controlled by a computer to solidify specific parts of the liquid surface. The platform is lowered and the cycle repeats, manufacturing the object layer by layer. Laser absorption happens a few tenths of a millimetre which corresponds to the thickness of each layer \citep{kruth1991material, pham1998comparison}.

Fused deposition modelling is another liquid-based AM technology. A jetting head, or nozzle, deposits the molten material onto a platform or on top of the previous layer. The material is usually plastic in the form of a thin filament. It is heated to just above its melting points, typically \SI{1}{\degreeCelsius} \citep{crump1992apparatus}, so that it cools down within \SI{0.1}{\second} \citep{kruth1991material}. The platform moves, and controlled by a computer, in the $xy$-plane, or left to right and front to back, to produce a layer. The jetting head can move in the $z$-axis, or up and down, to manufacture the next layer. In the original patent by \cite{crump1992apparatus}, the thickness can be as thin as 0.000\,1 inches (\SI{0.003}{\milli\metre}).

3D printing is a powder-based AM technology. A jetting head deposits a binding agent in droplets onto a bed of powder of ceramic, metal or polymer. The binding agent is cured via evaporation or heating which glues the powder particles together. The jetting head can move in the $xy$-plane and the object is manufactured layer by layer by moving the platform in the $z$-axis and renewing the powder using a roller. The binding agent must have a low viscosity so it can be deposited and may also be charged so that it can be deflected using an electric field for precise deposition \citep{sachs1990three}. The thickness of each layer is determined by the size of the droplets of the binding agent, which can be as small as \SI{15}{\micro\metre} in diameter \citep{sachs1990three}. \cite{sachs1990three} reported a tolerance of 0.001 inches (\SI{0.03}{\milli\metre}).

Selective laser sintering, electron beam melting and laser engineered net shaping are powder-based AM technology. Selective laser sintering is similar to 3D printing, but instead, a laser is used to sinter or fuse the powder particles in a chamber heated just below the melting point of the material \citep{wong2012review}. Various materials such as metals and plastics can be used \citep{wong2012review}. Electron beam melting is similar, but instead of a laser, an electron beam is used. This is done in a high vacuum chamber to avoid oxidation \citep{wong2012review}. In laser engineered net shaping, a powder bed is not used; the powder is deposited on the desired location and then melted using a laser, as shown in Figure \ref{fig:literature_lens}. This is a popular method to manufacture metal objects \citep{gibson2010additive}.

\begin{figure}
  \centering
  \includegraphics[width=0.7\textwidth]{../figures/literatureReview/literature_lens.png}
  \caption{Laser engineered net shaping. Reprinted from \citep{wong2012review} under the CC BY 3.0 license.}
  \label{fig:literature_lens}
\end{figure}

There are many more AM technologies but they can be found in numerous review literature. A comparison of the mentioned AM technologies available at the time was done by \cite{pham1998comparison, kim2008benchmark}. Factors such as material cost, mechanical properties and the resolution of the manufacturing were considered. There are also safety aspects to assess, for example, powder in powder-based methods can escape into the environment and the liquid used in stereolithography is toxic, sticky and has spilling risk \citep{kim2008benchmark}. This makes fused deposition modelling a popular choice and can be used in an office environment \citep{ngo2018additive}.

The strength of the manufactured object varies from geometry to geometry but also from direction to direction. Because the manufactured object is made layer by layer, the strength varies if the load was applied in the building direction (vertical) or the scanning direction (horizontal) \citep{kim2008benchmark}. Experimental results have shown that fused deposition modelling has superior strength in the scanning direction but weak in the building direction \citep{kim2008benchmark}.

The strongest manufacturing methods were found to be powder-based methods and stereolithography, however, they are slow and material costs are high \citep{kim2008benchmark}. Fused deposition modelling has low costs and high speeds but suffers from weak mechanical properties \citep{ngo2018additive}.

The materials available for each AM technology varies. The materials used in stereolithography is limited because of the use of liquids with photo-hardening properties \citep{ngo2018additive}. Fused deposition modelling is limited to plastics \citep{ngo2018additive}. Selective laser sintering and laser engineered net shaping can manufacture objects using metals such as aluminium alloys, steel, titanium and titanium alloys \citep{herzog2016additive}.

\subsection{Pre/Post Processing}

The blueprint of the object to be manufactured is called a computer-aided design (CAD) model. For it to be processed by an AM apparatus, the CAD model is converted to an STL file \citep{3d1989sterolithography, 3d2019what} which represent surfaces by a series of triangles, an example is shown in Figure \ref{fig:literature_stl}. STL stands for stereolithography but could also be called standard tessellation language \citep{wong2012review}. Some accuracy is lost here as the surface of the CAD model is represented approximately by triangles \citep{gibson2010additive}. The STL file is then sliced into layers \citep{jamieson1995direct, vatani2009enhanced} so that the AM apparatus knows what to build for each layer.

\begin{figure}
  \centering
  \includegraphics[width=0.9\textwidth]{../figures/literatureReview/literature_stl.png}
  \caption{An example of a CAD model (left) converted to an STL file (right). Republished with permission of Springer New York, from \cite{gibson2010additive}; permission conveyed through Copyright
Clearance Center, Inc.}
  \label{fig:literature_stl}
\end{figure}

When the AM object is manufactured, post-processing techniques can be done at this stage. For example, sanding may be done to smooth the surfaces \citep{gibson2010additive}. The manufactured object may be inspected for pores or defects by comparing the x-ray projection of the object with the CAD model \citep{lee2015compliance, villarraga2015assessing, kim2016inspection}.

As with any apparatus, regular maintenance is required \citep{bell2014maintaining}.

\subsection{Defects and Quality Control}

There are various discontinuities in AM. In fused deposition modelling, a staircase effect on the surface of the product arises from poor slicing methods of the CAD model \citep{weeren1995quality}. Internal voids can be formed due to insufficient material flow \citep{weeren1995quality}. Other factors which can cause defects include misalignment of the platform or nozzle, depletion of material and lack of adhesion due to low temperatures \citep{gunaydin2018common}.

There are also problems in the manufacturing of metal parts \citep{everton2016review}, for example, gas can become trapped during the manufacturing process forming gas pores in the manufactured object \citep{thijs2010study, tammas2015xct}. These gas pores can be \SIrange{5}{20}{\micro\metre} in diameter \citep{everton2016review}.

Layers may not fuse and form elongated pores. This can be fixed by increasing the energy of the beam but increasing it too much will cause evaporation of the AM part \citep{mumtaz2008high}. These pores can be \SIrange{50}{500}{\micro\metre} in size \citep{everton2016review} and can be observed using a scanning electron microscope as shown in Figure \ref{fig:literature_pores}.

Low wetting ability of the melt pool can cause balling which is where the sintered powder has poor contact on the existing layer causing spherical particles to form on the surface of the AM part \citep{li2012balling, gu2009balling}. The spherical particles can vary in size of \SIrange{10}{500}{\micro\metre} \citep{li2012balling}. The balling effect can be reduced by ensuring low oxygen content in the environment \citep{niu1999instability} and using higher energy beams \citep{gu2009balling}. Some examples are shown in Figure \ref{fig:literature_balling} using an electron scanning microscope.

Cracks can form due to extreme temperature changes and gradients \citep{mercelis2006residual, zaeh2010investigations}.

\begin{figure}
  \centering
  \includegraphics[width=0.99\textwidth]{../figures/literatureReview/literature_pores.png}
  \caption{A scanning electron microscope image of a) pores and b) elongated pores from an electron beam melting manufactured object. Reprinted from \cite{tammas2015xct} under the CC BY 4.0 license.}
  \label{fig:literature_pores}
\end{figure}

\begin{figure}
  \centering
  \includegraphics[width=0.7\textwidth]{../figures/literatureReview/literature_balling.png}
  \caption{A scanning electron microscope image of balling on a selective laser melting manufactured object. Reprinted by permission from Springer Nature: \cite{li2012balling}\textsuperscript{\textcopyright}.}
  \label{fig:literature_balling}
\end{figure}

The manufacturing process can be monitored, this is called online or in-situ process monitoring \citep{everton2016review}. The idea is that problems during the manufacturing process are found as soon as possible before the final product is spoiled \citep{cerniglia2015inspection}. Various methods are used for in-situ process monitoring, for example, a high-speed camera can be installed to capture the various wavelengths in the electromagnetic spectrum emitted by the melt pool \citep{berumen2010quality, craeghs2011online, lott2011design}. Various discontinuities and errors can be detected \citep{clijsters2014in} and be used to give feedback to the AM apparatus \citep{herzog2013method}. Other methods include measuring the surface using a laser \citep{cerniglia2015inspection} and using an infrared camera to measure the temperature of the melt pool \citep{rodriguez2012integration}.

\section{X-ray Computed Tomography}

XCT started its use in the medical field but the advancement of the technology saw its use in manufacturing and metrology, the science of measurement. Applications of XCT include the examination of acetabular hip prosthesis cups \citep{kourra2018computed}, skeletons \citep{appleby2014scoliosis}, batteries \citep{taiwo2017investigating} and materials \citep{zhang2016x, wang2017x}. XCT can be used to reverse engineer existing products and improvements can be fabricated using AM, for example, it was used for improving existing hollow engine valves \citep{cooper2015design}. However, the use of XCT in metrology is not yet firmly established compared to other methods of measurement \citep{thompson2016x}. This is because there are a lot of inconsistencies in the setup of XCT apparatuses and on controlling the sources of error.

\subsection{Concepts from the Medical Field}

The setup of XCT \citep{cormack1973reconstruction, hounsfield1973computerized, hounsfield1980computed} in the medical field involves the patient laying on a flatbed. An x-ray source and x-ray detector pair rotate around and translate along the patient to get readings of the x-rays after attenuating through the patient via different paths. X-ray beams were pencil beams in the early versions of XCT \citep{michael2001x}. To reduce scanning times, fan-shaped beams and arrays of detectors were used and they can move in a spiral fashion along and around the patient \citep{cierniak2011x}. These multiple x-ray readings can be used to reconstruct a representation of the patient in 3D \citep{zeng2010medical}. This is illustrated in Figure \ref{fig:literature_medicalct}.

\begin{figure}
  \centering
  \includegraphics[width=0.99\textwidth]{../figures/literatureReview/literature_medicalct.png}
  \caption{In medical XCT, a fan-shaped x-ray beam is emitted and attenuate through the patient and detected by a detector. The x-ray source and detector rotate around and translate along the patient. By collecting readings at different angles, the image of the patient can be reconstructed. Reprinted from \cite{michael2001x}. \textcopyright\ IOP Publishing. Reproduced with permission. All rights reserved.}
  \label{fig:literature_medicalct}
\end{figure}

The patient cannot be exposed to too much radiation, therefore the x-rays used are of low power which can cause noisy readings from the detector. The sources of noise are from the behaviour of the x-rays and the electronics in the detector \citep{yang2010noise}. In this realm of low signal to noise ratio, the noise has a compound Poisson element to it \citep{whiting2002signal, whiting2006properties}. Many reconstruction algorithms have been proposed to consider the compound Poisson noise \citep{elbakri2002statistical, elbakri2003efficient, elbakri2003statistical, lasio2007statistical, xie2008x}.

\subsection{Acquisition Process in Manufacturing}

In manufacturing and metrology, high power x-rays can be used in XCT because there is no consequence of the manufactured object absorbing the radiation. As a result, the XCT setup is different. The object is held by foam on a turntable and placed between an x-ray source and an x-ray detector. X-ray projections are taken while the object rotates. Typically, the x-ray is a cone-beam \citep{kruth2011computed}. This is illustrated in Figure \ref{fig:literature_xct}.

\begin{figure}
  \centering
  \includegraphics[width=0.75\textwidth]{../figures/literatureReview/literature_xct.png}
  \caption{The setup of XCT used in metrology. Reprinted from \cite{warnett2016towards} under the CC BY 3.0 license.}
  \label{fig:literature_xct}
\end{figure}

The acquisition process consists of the production of x-rays, x-rays attenuating the object, the detection of x-rays and the reconstruction process.

X-rays \citep{rontgen1896on} are produced in an x-ray tube, a diagram shown in Figure \ref{fig:literature_tube}. It consists of a vacuum tube containing a cathode and an anode. Electrons are fired from the cathode to the anode due to an electric potential. The cathode is usually tungsten and the anode contains a small amount of tungsten, molybdenum or copper \citep{sun2012overview}.

\begin{figure}
  \centering
  \includegraphics[width=0.8\textwidth]{../figures/literatureReview/literature_tube.png}
  \caption{An x-ray tube. Reprinted from \cite{michael2001x}. \textcopyright\ IOP Publishing. Reproduced with permission. All rights reserved.}
  \label{fig:literature_tube}
\end{figure}

The electrons can interact with the anode in many ways. The electrons can be deflected or decelerated due to the electric field from the nucleus of the target anode material. The energy lost by the electrons is emitted as bremsstrahlung radiation. The energy of the radiation depends on the potential difference in the x-ray tube, as this determines the energy of the fired electrons, and also the proton number of the anode target because this affects the electric field produced by the nucleus in the anode target \citep{sun2012overview}. Another interaction is when the electrons may collide with the nucleus in the anode target, exciting an inner shell electron and ionising it. This produces a vacancy in the electron shell and emits a photon when the excited electron drops down back to the ground state. This is known as characteristic radiation and the energy emitted is discrete and depends on the material in the anode target \citep{sun2012overview}.

The efficiency of an x-ray tube is poor. Over 99\% of the energy from electrons is converted to heat, the rest to x-rays \citep{kruth2011computed}.

Photons, making up the radiation, are emitted from the x-ray tube which can be modelled as a Poisson process \citep{whiting2006properties, cierniak2011x}. The rate of x-ray emission depends on the current, that is the rate of charge between the cathode and anode. Sources of energy of each photon come from bremsstrahlung radiation and characteristic radiation, making the distribution of x-ray photons energy a mix of continuous and discrete energies \citep{sun2012overview}. An example is shown in Figure \ref{fig:literature_spectrum}.

\begin{figure}
  \centering
  \includegraphics[width=0.65\textwidth]{../figures/literatureReview/literature_spectrum.png}
  \caption{An example of the distribution of energies a photon can have emitted from an x-ray tube. Bremsstrahlung and characteristic radiation contribute to the continuous and discrete components of the distribution. Reprinted from \cite{michael2001x}. \textcopyright\ IOP Publishing. Reproduced with permission. All rights reserved.}
  \label{fig:literature_spectrum}
\end{figure}

The scanned object is exposed to x-ray photons which undergo attenuation when interacting with the object in several ways \citep{cantatore2011introduction}. The object can absorb the photons via the photoelectric effect. In the photoelectric effect, a photon transfer all of its energy to a bounded electron and ejects it from the atom in the object \citep{millikan1916direct}. Photons can be scattered by the object by colliding inelastically with and transfers its energy to an electron. This process is known as Compton scattering \citep{compton1923quantum}. The photoelectric effect and Compton scattering cause several photons to be undetectable. If some of the photons avoid these processes, they are detected with their energy unaffected.

Beer's law simplifies these quantum mechanistic process. Suppose the x-ray beam with a rate of emission $I_0$ is mono-energetic and travels in a straight line in the $x$-axis. Let $\mu(x)$ be the attenuation coefficient of the object and the x-ray beam has a rate of emission $I$ after attenuation. A differential equation can be set up to model the decay of photons as it attenuates through the object such that
\begin{equation}
\dfrac{\diff I}{\diff x} = -I\mu(x)
\end{equation}
which can be solved
\begin{equation}
I = I_0\exp\left[\int_{x \in \text{path of photon}}-\mu(x)\diff x\right] \ .
\label{eq:beerLaw}
\end{equation}
However, the photoelectric effect and Compton scattering, thus the attenuation coefficient as well, depends on the energy of the photons \citep{elbakri2002statistical}. Therefore $\mu(x,E)$ should be made dependent on the energy of the photons \citep{cantatore2011introduction} and can cause some inaccuracies in Beer's law. In general, low energy photons are more likely to be absorbed and scattered than high energy photons, which increases the average energy of the detected photons \citep{sun2012overview}. This is called beam hardening.

After attenuation, the x-ray photons are detected by the x-ray detector. The detectors used in XCT are typically flatbed scanners made up of a scintillator material \citep{curran1953luminescence, greskovich1997ceramic} and photodiodes. The x-ray photons interact with the scintillator material and produce visible light pulses \citep{rossner1993conversion}. These pulses are detected by photodiodes and converted into an electrical signal \citep{nikl2006scintillation, ren2018tutorial}. The electrical signal can be a quantum counter, counting the number of photons detected, or an energy integrating detector, adding up all of the energies of each detected photon \citep{nikl2006scintillation, whiting2006properties, kruth2011computed, ren2018tutorial}. The electrical signals are subject to sampling and quantisation to store these signals as an image \citep{cierniak2011x}. This image is known as a projection.

Not all of the visible light pulses are detected by the photodiodes, thus not all the x-ray photons are detected. The ratio between the number of x-ray photons detected by the detector and the number of x-ray photons arriving at the detector is called the quantum efficiency \citep{cierniak2011x, ren2018tutorial}. This makes the detection a two-stage process, converting the x-ray photons into visible light which are then detected \citep{cierniak2011x}. There exist equipment which detects x-ray directly such as a xenon gas ionisation detector \citep{fuchs2000direct} but this is unrivalled by solid-state CT systems, such as scintillator-photodiodes detectors, which have a high quantum efficiency of about 98\% to 99.5\% \citep{hsieh2000investigation}.

Once projections of the object have been acquired at multiple angles, the reconstruction process can start. The objective of reconstruction is to estimate the attenuation coefficient of the object at each point in space $\mu(x,y,z)$ using the x-ray projections. This is done using the fact that the projections are based on the line integral of the attenuation coefficient along the path of photons. This problem was formed by \cite{radon1986on} as the `determination of functions from their integral values along certain manifolds'.

A number of reconstruction algorithms in XCT have been developed \citep{smith1990cone} such as the filtered back-projection \citep{brooks1976principles} and the FDK algorithm \citep{feldkamp1984practical}. Once the reconstruction has been done, the shape or surface can be extracted by the use of thresholding \citep{kruth2011computed}. There are many software packages available for the reconstruction stage of XCT \citep{reinhart2008industrial, sun2012overview}.

\subsection{Metrology in Practice}

XCT can be used to measure lengths and distances, making it useful for measuring the dimensions of AM objects internally and externally. \emph{Nikon} offer products and services for XCT including features such as direct comparison to the CAD model \citep{nikon2015microfocus, nikon2018mct225} and automated production line inspection \citep{nikon2015inline, nikon2018automated}.

As with a lot of measurement apparatus, calibration is required. In XCT, the scale of each voxel in the reconstruction can be obtained by using XCT on an object with pre-determined lengths, these are known as reference standards \citep{bartscher2007enhancement} but can have similar names. Reference standards can vary in geometry such as a sphere on a cylinder \citep{lifton2013application}, two spheres on a cylinder \citep{sun2016reference}, a cube with cut-outs \citep{kiekens2011test}, a hollow cylinder, a step-cylinder and a ball-bar \citep{bartscher2007enhancement}; the latter three are shown in Figure \ref{fig:literature_referenceStandards}.

\begin{figure}
  \centering
  \centerline{
    \begin{subfigure}[b]{0.49\textwidth}
      \includegraphics[width=\textwidth]{../figures/literatureReview/literature_test1.png}
      \caption{Hollow cylinder}
    \end{subfigure}
    \begin{subfigure}[b]{0.49\textwidth}
      \includegraphics[width=\textwidth]{../figures/literatureReview/literature_test2.png}
      \caption{Step-cylinder}
    \end{subfigure}
  }
  \begin{subfigure}[b]{0.6\textwidth}
    \includegraphics[width=\textwidth]{../figures/literatureReview/literature_test3.png}
    \caption{Ball-bar}
  \end{subfigure}
  \caption{Various reference standards: a) aluminium hollow cylinder, outer diameters \SI{30}{\milli\metre} and \SI{20}{\milli\metre}, b) aluminium step-cylinder with diameter \SI{300}{\milli\metre}, c) ceramic balls of diameter \SI{30}{\milli\metre} on a carbon fibre rod, the balls are separated by \SI{100}{\milli\metre}. Reprinted from \cite{bartscher2007enhancement}\textsuperscript{\textcopyright} with permission from Elsevier.}
  \label{fig:literature_referenceStandards}
\end{figure}

There are many variables in XCT and a lot of them have to be controlled, for example, XCT should be done in room temperature to avoid any thermal variation \citep{bryan1990international}, however, this can be hard to do when the x-ray tube is a heat source \citep{kruth2011computed}.

The potential difference and current of the x-ray tube can be adjusted to control the contrast and brightness of the x-ray projection. The exposure time is also a factor. These settings should be set high enough to avoid beam extinction but low enough that there is a contrast where less material is present \citep{kruth2011computed}.

The magnification can be modified by altering the distances between the x-ray tube, the object and the x-ray detector. Increasing the magnification increases the image resolution but can cause blurry images, this is the result of using an x-ray source with a finite spot size as shown in Figure \ref{fig:literature_magnification} \citep{kruth2011computed}. Larger spot sizes cause more blurry results, this is known as the penumbra effect \citep{kueh2016modelling}. However, spot sizes too small can produce concentrated heat \citep{welkenhuyzen2009industrial} and can damage the x-ray tube.

\begin{figure}
  \centering
  \includegraphics[width=0.7\textwidth]{../figures/literatureReview/literature_magnification.png}
  \caption{The magnification can be tuned by adjusting the distances between the x-ray source, the object and the x-ray detector. Because of a finite x-ray spot size, blurry effects are produced using a magnification too large. Reprinted from \cite{kruth2011computed}\textsuperscript{\textcopyright} with permission from Elsevier.}
  \label{fig:literature_magnification}
\end{figure}

There is also the question on how to orient the object on the turntable \citep{corcoran2016observations} as well as how many angles to use \citep{kruth2011computed}. More angles produce a more accurate reconstruction but require more acquisition time. Figure \ref{fig:literature_angles} shows an example of a reconstruction using various numbers of angles.

\begin{figure}
  \centering
  \includegraphics[width=0.7\textwidth]{../figures/literatureReview/literature_angles.png}
  \caption{A reconstruction when scanning three aligned balls using a different number of angular projections. Reprinted from \cite{kruth2011computed}\textsuperscript{\textcopyright} with permission from Elsevier.}
  \label{fig:literature_angles}
\end{figure}

All of the parameters of XCT discussed can be determined before the XCT process by use of simulations \citep{reisinger2011simulation, reiter2011simulation}, however, there may be inconsistencies. For example, even though the target material of the anode and power is specified, the energy spectrum can still vary \citep{stumbo2004direct}.

Problems can occur in the detector, for example, pixels in the acquired projection can be defective or dead \citep{brettschneider2014spatial}, the panel structure of the detector can be observed \citep{yang2009evaluation} and the cone-beam appears as a spot. The x-ray spot could be fitted by using a mixture of a Gaussian spot and a uniform spot \citep{kueh2016modelling}. Another problem is that there exist spatially correlated noise within a projection which can be detected experimentally \citep{sun2016characterisation} as well as a correlation between acquisitions, known as image lag \citep{yang2009evaluation}. Precautions can be taken to reduce the impact from image lag such as waiting for 20 minutes between acquisitions \citep{yang2010noise}.

Errors due to beam hardening can occur. Low energy photons are more likely to be absorbed or scattered, which causes a few millimetres of the surface of the object to absorb or scatter more photons than the interior. This can cause artefacts \citep{sun2016applications} such as flat surfaces to be barrelled and edges rounded off \citep{kruth2011computed}. Beam hardening can be tackled by eliminating the low energy photons by placing a filter, a thin metal plate, in front of the x-ray tube \citep{welkenhuyzen2009industrial}, for example, copper. Figure \ref{fig:literature_hardening} shows an example of reconstruction with and without a filter. Without the filter, the interior of the object appears less dense than it should be.  A filter reduces the rate of photon emission but this can be compensated by increasing the exposure time \citep{kruth2011computed}. Early reconstruction algorithms ignored beam hardening but modern methods can take beam hardening into account \citep{elbakri2001statistical, sun2016applications}.

\begin{figure}
  \centering
    \begin{subfigure}[b]{0.4\textwidth}
      \includegraphics[width=\textwidth]{../figures/literatureReview/literature_hardening_noFilter.png}
      \caption{No filter}
    \end{subfigure}
    \begin{subfigure}[b]{0.4\textwidth}
      \includegraphics[width=\textwidth]{../figures/literatureReview/literature_hardening_filter.png}
      \caption{Al/Cu filter}
    \end{subfigure}
  \caption{A reconstruction of a hollow cylinder, outer diameter \SI{6.0}{\milli\metre} and inner diameter \SI{0.6}{\milli\metre}. In a), no filter was used. In b) a filter was placed in front of the x-ray tube. Reprinted from \citep{kruth2011computed}\textsuperscript{\textcopyright} with permission from Elsevier}
  \label{fig:literature_hardening}
\end{figure}

The most common reconstruction method is the FDK \citep{feldkamp1984practical} algorithm because it caters for cone beams. However, it assumes a circular trajectory from the source which can cause artefacts if the trajectory is not circular \citep{sun2016applications}.

\subsection{Latest Research}

The most common use of XCT in AM is the investigation of pores in the manufactured object \citep{thompson2016x}. Pores can be classified as defects if the pores are larger than some volume threshold. This threshold controls the probability of the detection of defects \citep{gandossi2010probability, amrhein2014characterization}.

The porosity is defined by dividing the volume of all of the pores by the volume of solid material \citep{taud2005porosity}. This can be used to quantified the material's strength and can be measured accurately using Archimedes' method \citep{spierings2011comparison}. Studies have been done to link porosity to stress concentration \citep{leuders2015fatigue, siddique2015computed, carlton2016damage} and it was found the location of the pores is a good predictor of fatigue strength \citep{leuders2015fatigue}. XCT can be used to measure porosity and has an advantage over Archimedes' method because the location of the pores can be visualised in XCT. An example of visualising pores is shown in Figure \ref{fig:literature_pores3D} and the pores can be compared to the CAD model \citep{lee2015compliance, villarraga2015assessing, kim2016inspection}. In addition to pores, any surface deviation can be measured by aligning the reconstruction with the CAD model and measuring any discrepancies \citep{lee2015compliance, villarraga2015assessing, kim2016inspection}, an example is shown in Figure \ref{fig:literature_warnett}.

\begin{figure}
  \centering
      \begin{subfigure}[b]{0.49\textwidth}
      \includegraphics[width=\textwidth]{../figures/literatureReview/literature_pores3D1.png}
      \caption{Reconstruction}
    \end{subfigure}
    \begin{subfigure}[b]{0.49\textwidth}
      \includegraphics[width=\textwidth]{../figures/literatureReview/literature_pores3D2.png}
      \caption{Sample from the reconstruction}
    \end{subfigure}
    \caption{The reconstruction can show pores in the manufactured object. b) shows the reconstructed samples from the blue cubes in a). Reprinted from \cite{tammas2015xct} under the CC BY 4.0 license.}
    \label{fig:literature_pores3D}
\end{figure}

\begin{figure}
  \centering
      \begin{subfigure}[b]{0.49\textwidth}
      \includegraphics[width=\textwidth]{../figures/literatureReview/literature_warnett1.png}
      \caption{External}
    \end{subfigure}
    \begin{subfigure}[b]{0.49\textwidth}
      \includegraphics[width=\textwidth]{../figures/literatureReview/literature_warnett2.png}
      \caption{Internal}
    \end{subfigure}
    \caption{The reconstruction was aligned and compared to the CAD model. The surface heat map shows the surface deviation (or `variance' in the literature) externally (a) and internally (b). Reprinted from \cite{warnett2016towards} under the CC BY 3.0 license.}
    \label{fig:literature_warnett}
\end{figure}

One of the disadvantages of XCT is that it is a slow process. XCT is not an instantaneous process so progress bars are usually featured in XCT marketing such as \cite{nikon2015inline}'s inline quality control. The reconstruction can take between 5 minutes to several hours \citep{warnett2016towards}. More angular projections would take more time but will improve the accuracy of the reconstruction \citep{kruth2011computed}. \cite{warnett2016towards} improved the speed of XCT by sacrificing the accuracy of the reconstruction. This was done by placing the object on a conveyor belt surrounded by multiple x-ray source and detector pairs as shown in Figure \ref{fig:literature_conveyor}. Fewer angular projections were taken but they can be obtained in one go, speeding up the process.

\begin{figure}
  \centering
  \includegraphics[width=0.7\textwidth]{../figures/literatureReview/literature_conveyor.png}
  \caption{XCT can be done on a conveyor belt surrounded by x-ray source and detector pairs. Reprinted from \cite{warnett2016towards} under the CC BY 3.0 license.}
  \label{fig:literature_conveyor}
\end{figure}

Instead of reconstructing the object, the analysis can be done on the projections itself, or in projection space, by comparing it to a simulated projection produced by a software called \emph{aRTist} \citep{bellon2007artist, jaenisch2008artist, bellon2012radiographic}. It can simulate projections of the object given the specifications of the CT apparatus, such as the x-ray source and the x-ray detector, and the CAD of the object \citep{bellon2011simulation, deresch2012simulating}.

An algorithm was developed to adjust the parameters of the simulation as well as aligning it so that it fits with the x-ray acquisition \citep{brierley2018optimized}. However, it is very complicated as it is optimising over a very large dimensional space \citep{brierley2018optimized}. Studies have been conducted comparing simulated projections with each other, one with defects and the other without, by looking at the contrast to noise ratio of the defects. This is done at various projection angles as part of a large optimisation problem \citep{brierley2018optimized}. Another method is to use machine learning methods to classify defects from a projection \citep{rale2009comparison}.

It is however inevitable that accuracy is lost from the transition from reconstruction space to projection space, for example, in the diagnostic of pneumonia, a CT scan has superior performance compared to a chest radiograph \citep{hayden2009chest}.


\chapter{Data Collection}

\chapter{Compound Poisson}
The grey value of each pixel in the detector can be modelled as a random variable, due to the random behaviour of photons being produced, interacting with the test sample and the scintillator in the detector. By modelling using a random variable, the uncertainty can be quantified and considered when conducting inference about any detected defects.

The compound Poisson distribution is studied here because of the compound Poission-like behaviour from the detection of photons. It is defined by defining a latent variable $Y\sim\poisson(\lambda)$ with probability mass function (p.m.f.)
\begin{equation}
  \prob(Y=y)=\euler^{-\lambda}\frac{\lambda^y}{y!} \quad \text{for }y=0,1,2,\cdots
\end{equation}
where $\lambda>0$ is the Poisson rate parameter. Let $U_i$ be some independent and identically distributed (i.i.d.) latent random variables with probability density function (p.d.f.) $p_U(u)$ for $i=1,2,3,\dotdotdot$. Let $X$ be a compound Poisson random variable where
\begin{equation}
  X|Y = \sum_{i=1}^{Y}U_i \ .
  \label{eq:compoundPoisson_X|Y}
\end{equation}
The p.d.f.~of $X$ can be obtained by marginalising the joint p.d.f.
\begin{equation}
  p_X(x)=\sum_{y=0}^\infty p_{X|Y}(x|y)\prob(Y=y) \quad\text{for }x\geqslant 0
  \ .
\end{equation}
It should be noted that $X=0$ if and only if $Y=0$ with probability $\prob(X=0)=\euler^{-\lambda}$. Then $X$ has probability mass at $X=0$ and probability density at $X>0$ which results in the p.d.f.
\begin{equation}
  p_X(x) = 
  \begin{cases}
    \delta(x) \euler^{-\lambda}  & \text{ for } x=0 \\ 
    \sum_{y=1}^\infty p_{X|Y}(x|y)\euler^{-\lambda}\frac{\lambda^y}{y!} \quad\text{for } & \text{ for } x>0
  \end{cases}
\end{equation}
where $\delta(x)$ is the Dirac delta function.

This chapter starts with a literature review on the compound Poisson distribution, how it is derived from the behaviour of photons, how its likelihood is evaluated and methods for fitting onto data. A model was proposed for the pixel's grey values and the expectation-maximization (EM) algorithm was implemented to fit the model onto data. It was found that for high photon rate, there were identifiability issues. The chapter is concluded on a discussion on why the EM algorithm failed.

\section{Literature Review}

\subsection{Compound Poisson in X-ray Detection}

In an x-ray tube, photons are emitted as a Poisson process and each photon has some random energy due to bremsstrahlung and characteristic radiation. This shares similarities to the compound Poisson. Let $Y$ be the number of photons emitted for some time exposure $\tau$, then $Y\sim\poisson(\lambda)$. Each photon is assumed to be i.i.d.~with energy $U_i$ for $i=1,2,3,\cdots$. $U_i$ has p.d.f.~$p_U(u)$. The random variables discussed here covers all the latent variables in the compound Poisson.

Photons emitted from the x-ray tube undergo attenuation when propagating through the test sample. Assuming no beam hardening, some photons are either absorbed or scattered, making them undetectable. Scattered photons may be detected but it is very rare. Attenuated photon energy remains unaffected so attenuation decreases the parameter $\lambda$ and the amount it decreases by depends on the attenuation coefficient of the material and the amount of material the x-ray attenuates. The parameters of the random variable $U_i$ remains unchanged because the energy of each photon remains the same after attenuation, assuming no beam hardening.

When the photons interact with the scintillator in the detector, they are converted into visible light. The visible light photons are then detected and converted into a a grey value. A quantum counter set the digital signal to be linear with to the number of photons detected \citep{whiting2006properties}. Let $X$ be the grey value observed, then
\begin{equation}
X = bY + \epsilon
\end{equation}
where $\epsilon\sim\normal(a,\kappa)$, $b$ and $a$ are some constant and $\kappa$ is the variance of electronic noise. In a quantum counter, the mean and variance of the grey value are
\begin{equation}
\expectation\left[X\right] = b\lambda + a
\end{equation}
and
\begin{equation}
\variance\left[X\right] = b^2\lambda + \kappa
\end{equation}
respectively. By eliminating $\lambda$
\begin{equation}
\variance\left[X\right] = b\expectation\left[X\right]+\kappa-ab \ ,
\end{equation}
a linear relationship between the variance and expectation of the grey value is obtained.

An energy integrating detector records the grey value which is linear to the energy detected \citep{whiting2006properties}. The grey value $X$ is
\begin{equation}
X|Y = \sum_{i=1}^Y U_i + \epsilon \ .
\end{equation}
This is the compound Poisson with Normal noise added to it. The scale factor $b$ is not included as this can be absorbed into $U$. Using the result that $\expectation\left[X\right]=\expectation\expectation\left[X|Y\right]$ and $\variance\left[X\right] = \variance\expectation\left[X|Y\right] + \expectation\variance\left[X|Y\right]$, the mean and variance of the grey value are
\begin{equation}
\expectation\left[X\right] = \lambda\expectation\left[U\right]+a
\end{equation}
and
\begin{equation}
\variance\left[X\right] = \lambda \expectation\left[U^2\right]+\kappa
\end{equation}
respectively. Eliminating $\lambda$ obtains
\begin{equation}
\variance\left[X\right] = \dfrac{\expectation\left[U^2\right]}{\expectation\left[U\right]} \expectation\left[X\right] + \kappa - a\dfrac{\expectation\left[U^2\right]}{\expectation\left[U\right]} \ .
\end{equation}
By assuming no beam hardening, $\expectation\left[U\right]$ and $\expectation\left[U^2\right]$ remains constant, thus there is a linear relationship between the variance and expectation of the grey value. There are other types of detection schemes \citep{whiting2006properties} but it shall be not be considered here.

Experiments have been done to verify the compound Poisson nature of the detector. This was done by investigating the variance of radiographs of air \citep{hsieh2015compound} and a polyethylene cylinder \citep{yang2009evaluation} \citep{yang2010noise} at different voltages and powers. It was found there were 2 components in the noise, one was signal dependent and comes from the compound Poisson, the other was signal independent and may be electronic noise. The electronic noise can be modelled as Normally distributed \citep{xu2009electronic}.

\subsection{Moment Generating Function}

Returning back to the compound Poisson with no electronic noise $X|Y = \sum_{i=1}^{Y}U_i$. Let the moment generating function (m.g.f.) of $X$ be
\begin{equation}
  M_X(\theta)=\expectation\left[\euler^{X\theta}\right]
  \ .
\end{equation}
This can be computed using the result for conditional expectations $M_X(\theta)=\expectation\expectation\left[\euler^{X\theta}|Y\right]$. Using the definition of $X|Y$ in Equation \eqref{eq:compoundPoisson_X|Y}, then
\begin{align}
  M_X(\theta)&=\expectation\expectation\left[\exp\left(\theta U_1 + \theta U_2 + \dotdotdot + \theta U_Y\right)|Y\right]
  \\
  &=\expectation\expectation\left[\euler^{\theta U_1}\cdot\euler^{\theta U_2}\cdot\dotdotdot\cdot\euler^{\theta U_Y}|Y\right]
  \ .
\end{align}
But because $U_i$ for $i=1,2,3,\dotdotdot$ are i.i.d., then each $U_i$ has a common m.g.f.~$M_U(\theta)=\expectation\left[\euler^{U\theta}\right]$, then
\begin{align}
  M_X(\theta)&=\expectation\left(
    \expectation\left[\euler^{\theta U_1}|Y\right]\cdot
    \expectation\left[\euler^{\theta U_2}|Y\right]\cdot
    \dotdotdot \cdot
    \expectation\left[\euler^{\theta U_Y}|Y\right]
  \right)
  \\
  &=\expectation\left[\left(M_U(\theta)\right)^Y\right]
  \\
  &=\expectation\left[\euler^{Y\ln(M_U(\theta))}\right]
  \\
  & = M_Y\left(\ln(M_U(\theta)\right)
\end{align}
where $M_Y(\theta)$ is the m.g.f.~of $Y$. It can be shown that the m.g.f.~of $Y$ is
$
  M_Y(\theta)=\expectation\left[\euler^{Y\theta}\right]=
  \exp
  \left[
    \lambda
    \left(
      \euler^\theta-1
    \right)
  \right]
$
then
\begin{equation}
  M_X(\theta)=
  \exp\left[
    \lambda
    \left(
      M_U(\theta)-1
    \right)
  \right]
  \ .
\end{equation}

Moments of $X$ can be obtained from the m.g.f.~by differentiating it and setting it to zero
\begin{align}
  M_X'(\theta)&=\exp\left[\lambda\left(M_U(\theta)-1\right)\right]\cdot\lambda M_U'(\theta) \\
  &=M_X(\theta)\lambda M_U'(\theta)
\end{align}
then
\begin{equation}
  \expectation\left[X\right]=\lambda\expectation\left[U\right]
  \ .
\end{equation}
Conducting the same procedure
\begin{equation}
  M_X''(\theta)=M_X'(\theta)\lambda M_U'(\theta)+M_X(\theta)\lambda M_U''(\theta)
  \ ,
\end{equation}
the variance can be obtained
\begin{align}
  \variance\left[X\right]&=M_X''(0)-\left[M_X'(0)\right]^2
  \\
  &=M_X'(0)\lambda M_U'(0)+M_X(0)\lambda M_U''(0)-\left[\lambda\expectation\left[U\right]\right]^2
  \\
  &=\lambda^2 \left(\expectation\left[U\right]\right)^2+\lambda \expectation\left[U^2\right]-\left[\lambda\expectation\left[U\right]\right]^2
\end{align}
resulting in
\begin{equation}
  \variance\left[X\right] = \lambda\expectation\left[U^2\right]
  \ .
\end{equation}
This can be extended for higher moments.

Because the m.g.f.~can be written down in closed form, so is the characterisation function or Fourier transform of the p.d.f.~of $X$. By obtaining an empirical version of $p_U(u)$, moments of $X$ can be estimated using the m.g.f.~and the p.d.f.~can be estimated by using fast Fourier transform on the empirical characteristic function \citep{whiting2006properties}.

\subsection{Generalised Linear Model}

A special case of the compound Poisson is when $U\sim\gammaDist\left(\alpha,\beta\right)$ where $\alpha>0$ is the gamma shape parameter and $\beta>0$ is the gamma rate parameter. The special case is used for example in \cite{xu2009electronic}. This is known as the compound Poisson-gamma distribution and denoted as $X\sim\CPoisson(\lambda,\alpha,\beta)$ and has p.d.f.
\begin{equation}
  p_X(x) = 
  \begin{cases}
    \delta(x) \euler^{-\lambda} & \text{ for } x=0 \\ 
    \sum_{y=1}^{\infty}\frac{\beta^{y\alpha}}{\Gamma(y\alpha)}x^{y\alpha-1}\euler^{-\beta x}\euler^{-\lambda}\frac{\lambda^y}{y!} & \text{ for } x>0
  \end{cases}
  \ .
  \label{eq:compoundPoisson_pdf}
\end{equation}
The conditional distribution can be shown to be $X|Y\sim\gammaDist\left(Y\alpha,\beta\right)$.

It can be shown that the compound Poisson-gamma distribution is in the exponential family for fixed $\alpha$ \citep{jorgensen1987exponential}. To show this, it is useful to reparametrize the compound Poisson-gamma distribution using the following:
\begin{equation}
  p=\frac{2+\alpha}{1+\alpha}
  \ ,
\end{equation}
\begin{equation}
  \mu=\frac{\lambda\alpha}{\beta}
  \ ,
\end{equation}
\begin{equation}
  \phi = \frac{\alpha+1}{\beta^{2-p}(\lambda\alpha)^{p-1}}
  \ .
\end{equation}
The parameters $p$, $\mu$ and $\phi$ are called the index, mean and dispersion parameters respectively. It can be shown that $1<p<2$, $\mu>0$ and $\phi>0$.

By rearranging the parameters to get
\begin{equation}
  \lambda=\frac{\mu^{2-p}}{\phi(2-p)}
\end{equation}
\begin{equation}
  \alpha=\frac{2-p}{p-1}
\end{equation}
\begin{equation}
  \beta=\frac{1}{\phi(p-1)\mu^{p-1}}
\end{equation}
and substituting it into Equation \eqref{eq:compoundPoisson_pdf}, the p.m.f.~at zero can be shown to be
\begin{equation}
  \prob(X=0) = \exp
  \left[
      -\frac{\mu^{2-p}}{\phi(2-p)}
  \right]
\end{equation}
and the p.d.f.~for $x>0$ is
\begin{multline}
  p_X(x) = \sum_{y=1}^{\infty}
  \left[
    \frac{1}{\phi(p-1)\mu^{p-1}}
  \right]^{y\alpha}
  \frac{1}{\Gamma(y\alpha)}
  x^{y\alpha-1}
  \exp\left[
      -\frac{x}{\phi(p-1)\mu^{p-1}}
  \right]
  \\
  \exp\left[
      -\frac{\mu^{2-p}}{\phi(2-p)}
  \right]
  \left[
    \frac{\mu^{2-p}}{\phi(2-p)}
  \right]^y
  \frac{1}{y!}
  \ .
\end{multline}
Tidying up the equation
\begin{multline}
  p_X(x) = 
  \exp\left[
    \frac{1}{\phi}\left(x\frac{\mu^{1-p}}{1-p}-\frac{\mu^{2-p}}{2-p}\right)
  \right]
  \frac{1}{x}
  \\
  \sum_{y=1}^{\infty}\frac{x^{y\alpha}\mu^{y[2-p-\alpha(p-1)]}}{\phi^{y(1+\alpha)}(p-1)^{y\alpha}(2-p)^yy!\Gamma(y\alpha)}
  \ .
\end{multline}
To simplify further, it should be noted that $2-p-\alpha(p-1) = 2-p - \frac{2-p}{p-1}(p-1) =0$ so that
\begin{equation}
  p_X(x) = 
  \exp\left[
    \frac{1}{\phi}
    \left(
      x\frac{\mu^{1-p}}{1-p}-\frac{\mu^{2-p}}{2-p}
    \right)
  \right]
  \frac{1}{x}
  \sum_{y=1}^{\infty}W_y(x,p,\phi)
\end{equation}
where
\begin{equation}
  W_y(x,p,\phi)=\frac{x^{y\alpha}}{\phi^{y(1+\alpha)}(p-1)^{y\alpha}(2-p)^yy!\Gamma(y\alpha)}
  \ .
\end{equation}
This is in the form of a generalised linear model \citep{nelder1972generalized, nelder1972generalized_2, mccullagh1984generalized} for fixed $p$ or $\alpha$ because the above is in the form of a distribution in the dispersive exponential family. Parameter estimation then can be done via the generalised linear model framework and can be extended to include linear mixed models \citep{zhang2013likelihood}. These has applications in for example insurance claim data \citep{jorgensen1994fitting, smyth2002fitting}.

Estimating $p$ is difficult and various methods were discussed by \cite{zhang2013likelihood}. One easy way is to estimate $\mu$ and $\phi$ on a grid of $p$'s and then select the $p$ which maximises the likelihood \citep{dunn2005series}.

One special property of the compound Poisson-gamma distribution is that it is in the Tweedie dispersion exponential family \citep{jorgensen1987exponential}. It has a special variance mean relationship
\begin{equation}
  \variance[X] = \phi \mu^p
\end{equation}
where, as a reminder, $1<p<2$. This can be derived using the partition function $Z$. Let
\begin{equation}
  \theta = \frac{\mu^{1-p}}{1-p}
\end{equation}
and
\begin{equation}
  \ln Z = \frac{1}{\phi(2-p)\theta^\alpha(1-p)^\alpha} \ ,
\end{equation}
then
\begin{equation}
  \variance[X] = \phi^2 \frac{\partial^2\ln Z}{\partial\theta^2} \ .
\end{equation}

\subsection{Method of Moments}

The method of moments is a simpler method to estimate the parameters of a compound Poisson-gamma random variable. The m.g.f.~of the of $X$ is
\begin{equation}
  M_X(\theta)=\exp\left[\lambda\left(\left(\frac{\beta}{\beta-\theta}\right)^{\alpha}-1\right)\right]
\end{equation}
and moments can be obtained from it such as
\begin{equation}
  \expectation\left[X\right]=\frac{\alpha\lambda}{\beta}
\end{equation}
\begin{equation}
  \variance\left[X\right]=\frac{\alpha(\alpha+1)\lambda}{\beta^2}
  \label{eq:compoundPoisson_variance}
\end{equation}
and
\begin{equation}
  \expectation\left[(X-\expectation[X])^3\right] = \frac{\alpha(\alpha+1)(\alpha+2)\lambda}{\beta^3}
  \ .
\end{equation}

Suppose $\widehat{\mu}$ is an estimator of $\expectation[X]$ and $\widehat{\mu}_j$ is an estimator of $\expectation\left[\left(X-\expectation[X]\right)^j\right]$ for $j=2,3$. Then the estimators
\begin{equation}
  \widehat{\lambda}=\frac{\widehat{\mu}^2\widehat{\mu}_2}{\left(2\widehat{\mu}_2^2-\widehat{\mu}_3\widehat{\mu}\right)}
\end{equation}
\begin{equation}
  \widehat{\alpha}=\frac{\widehat{\mu}_3\widehat{\mu}-2\widehat{\mu}_2^2}{\widehat{\mu}_2^2-\widehat{\mu}\widehat{\mu}_3}
\end{equation}
\begin{equation}
  \widehat{\beta}=\frac{\widehat{\mu}\widehat{\mu}_2}{\widehat{\mu}\widehat{\mu}_3-\widehat{\mu}_2^2}
\end{equation}
are method of moments estimators of $\lambda$, $\alpha$ and $\beta$ respectively \citep{withers2011compound}. These estimators suffer because estimation is not done through the sufficient statistics and can be negative, this is a problem because the parameters do not take non-positive values.

\subsection{Normal Approximation}

The evaluation of the density of a compound Poisson-gamma distribution is useful so that the likelihood can be obtained. The likelihood then can be used to find, for example, maximum likelihood estimators. A problem occurs when dealing with the infinite sum in the p.d.f.~because it cannot be simplified. There are a number of approximations or computational methods to evaluate the p.d.f.~such as Fourier inverting the characteristic function \citep{dunn2008evaluation}, using the saddlepoint approximation \citep{daniels1954saddlepoint} or cleverly sum over certain terms in the infinite sum \citep{dunn2005series}. Monte Carlo methods can be used to evaluate the p.d.f.~by simulating compound Poisson-gamma random variables.

The m.g.f.~of $X$ is
\begin{equation}
M(\theta)=\exp\left[\lambda\left(\left(\frac{\beta}{\beta-\theta}\right)^{\alpha}-1\right)\right]
\ .
\end{equation}
It provides a starting point to what limiting distributions the compound Poisson-gamma distribution converges to for large parameters. These limiting distributions then can be used to approximate the p.d.f.~of a compound Poisson-Gamma random variable. The m.g.f.~of $X$ itself it not useful because when considering $\lambda\rightarrow\infty$ or $\alpha\rightarrow\infty$ then $M_X(\theta)\rightarrow\infty$. Also for $\beta\rightarrow\infty$, then $M_X(\theta)\rightarrow 1$ which is not useful either.

The compound Poisson-gamma random variable $X$ can be standardised to obtain useful limiting results from the m.g.f. Let
\begin{equation}
  Z = \frac{X-\expectation[X]}{\sqrt{\variance[X]}}
  \ ,
\end{equation}
then using previous results
\begin{equation}
  Z = bX+a
\end{equation}
where
\begin{equation}
  b = \frac{\beta}{\sqrt{\alpha(\alpha+1)\lambda}}
\end{equation}
and
\begin{equation}
  a = -\sqrt{\frac{\alpha\lambda}{\alpha+1}}
  \ .
\end{equation}

The m.g.f.~of $Z$ is then
\begin{align}
  M_Z(\theta)&=\expectation\left[\euler^{Z\theta}\right]
  \\
  &=\expectation\left[\euler^{(bX+a)\theta}\right]
  \\
  &=\euler^{a\theta}M_X(b\theta)
  \ .
\end{align}
Substituting in the values
\begin{align}
  M_Z(\theta)&=
  \exp\left(
      -\theta\sqrt{\frac{\alpha\lambda}{\alpha+1}}
    \right)
  \exp\left[
    \lambda
    \left(
      \left(
        \frac{\beta}{\beta-\frac{\beta\theta}{\sqrt{\alpha(\alpha+1)\lambda}}}
      \right)^\alpha
      -1
    \right)
  \right]
  \\
  &=
  \exp\left(
      -\theta\sqrt{\frac{\alpha\lambda}{\alpha+1}}
    \right)
  \exp\left[
    \lambda
    \left(
      \left(
        \frac{\sqrt{\alpha(\alpha+1)\lambda}}{\sqrt{\alpha(\alpha+1)\lambda}-\theta}
      \right)^\alpha
      -1
    \right)
  \right]
\end{align}
or in a different form
\begin{equation}
	M_Z(\theta)=
	\exp\left(
	    -\theta\sqrt{\frac{\alpha\lambda}{\alpha+1}}
    \right)
	\exp\left[
		\lambda
		\left(
			\left(
				1-\frac{\theta}{\sqrt{\alpha(\alpha+1)\lambda}}	
			\right)^{-\alpha}
			-1
		\right)
	\right]
	\ .
\end{equation}

The above form is useful so that the binomial expansion can be conducted, that is
\begin{multline}
	\left(
		1-\frac{\theta}{\sqrt{\alpha(\alpha+1)\lambda}}
	\right)^{-\alpha}
	=
	1+
	\sum_{r=1}^\infty \frac{\theta^r\prod_{s=1}^r(\alpha+s-1)}{(\alpha(\alpha+1)\lambda)^{r/2}r!}
	\\
	\text{for }\frac{\|\theta\|}{\sqrt{\alpha(\alpha+1)\lambda}}<1 \ .
\end{multline}
Substituting in the binomial expansion
\begin{equation}
	M_Z(\theta)=
	\exp\left(
	    -\theta\sqrt{\frac{\alpha\lambda}{\alpha+1}}
    \right)
	\exp\left[
		\lambda
		\sum_{r=1}^\infty \frac{\theta^r\prod_{s=1}^r(\alpha+s-1)}{(\alpha(\alpha+1)\lambda)^{r/2}r!}
	\right]
	\ .
\end{equation}
Writing in full the $r=1,2$ terms
\begin{multline}
	M_Z(\theta)=
	\exp\left(
	    -\theta\sqrt{\frac{\alpha\lambda}{\alpha+1}}
    \right)
	\exp\left[
		\theta\sqrt{\frac{\alpha\lambda}{\alpha+1}}
		+\frac{\theta^2}{2}
	\right.
	\\
	\left.
		+\lambda\sum_{r=3}^\infty \frac{\theta^r\prod_{s=1}^r(\alpha+s-1)}{(\alpha(\alpha+1)\lambda)^{r/2}r!}
	\right]
\end{multline}
and a term cancels out to get
\begin{equation}
	M_Z(\theta)=
	\exp\left[
		\frac{\theta^2}{2}
		+\sum_{r=3}^\infty
		\frac
			{\theta^r\prod_{s=1}^r(\alpha+s-1)}
			{(\alpha(\alpha+1))^{r/2}r!}
		\lambda^{1-r/2}
	\right]
	\ .
\end{equation}

For large $\lambda$
\begin{equation}
	\lim_{\lambda\rightarrow\infty}M_Z(\theta)=\exp\left[\frac{\theta^2}{2}\right]
\end{equation}
which is the same as the m.g.f.~of a standard Normal variable, therefore
\begin{equation}
	\lim_{\lambda\rightarrow\infty}Z\sim\normal(0,1) \ .
\end{equation}
This should make sense as for high $\lambda$, the Poisson random variable has a high expectation, increasing the number of gamma random variables in a summation. Increasing the number of terms in a summation will trigger the central limit theorem.

Next for high $\alpha$, it should be noted that
\begin{equation}
  \lim_{\alpha\rightarrow\infty}
  \frac{
    \prod_{s=1}^r(\alpha+s-1)
  }
  {
    \alpha(\alpha+1))^{r/2}
  }
  =
  \frac{
    \prod_{s=1}^r\alpha
  }
  {
    \alpha^{r}
  }
\end{equation}
so that
\begin{equation}
  \lim_{\alpha\rightarrow\infty}
  \frac{
    \prod_{s=1}^r(\alpha+s-1)
  }
  {
    (\alpha(\alpha+1))^{r/2}
  }
  = 1
  \ .
\end{equation}
As a result
\begin{equation}
  \lim_{\alpha\rightarrow\infty}
  M_Z(\theta)=
  \exp\left[
    \frac{\theta^2}{2}
    +\sum_{r=3}^\infty
    \frac{\theta^r}{r!}
    \lambda^{1-r/2}
  \right]
  \ .
\end{equation}
However this shows that taking the limit $\alpha\rightarrow\infty$ is not enough to get a Normal limiting distribution. For a Normal limiting distribution, the limit must be accompanied with the limit $\lambda\rightarrow\infty$, that is
\begin{equation}
  \lim_{\lambda\rightarrow\infty}\lim_{\alpha\rightarrow\infty}Z\sim\normal(0,1)
  \ .
\end{equation}

Finally it should be noted that $M_Z(\theta)$ is independent of $\beta$. Thus $\beta$ will have no effect on the convergence to a Normal limiting distribution.

The above results justify the use of the approximation
\begin{equation}
	X\sim\normal\left(\frac{\lambda\alpha}{\beta},\frac{\lambda\alpha(\alpha+1)}{\beta^2}\right)
\end{equation}
for large $\lambda$. The limiting case where $\lambda\rightarrow 0$, $\alpha\rightarrow 0$ and $\beta\rightarrow 0$ will not be discussed here.

\subsection{Saddlepoint Approximation}

The saddlepoint approximation \citep{daniels1954saddlepoint} gives an approximate solution to inverting the Laplace transformation of the m.g.f.~giving the p.d.f. Inverting the Fourier transformation of the characteristic function also gives the p.d.f.~using computational methods \citep{dunn2008evaluation}. The saddlepoint approximation will be studied here.

The m.g.f.~of a random variable $X$ is defined to be
\begin{equation}
  M_X(\theta)=\expectation[\euler^{\theta X}]
  =
  \int_{-\infty}^{\infty}\euler^{x\theta} p_X(x) \diff x
  \ .
\end{equation}
For a given m.g.f., the saddlepoint approximation \citep{daniels1954saddlepoint, butler2007saddlepoint} finds an approximate $p_X(x)$. The saddlepoint approximation is given as
\begin{equation}
  p_X(x)\approx\left(2\pi K_X''(s)\right)^{-1/2}\exp\left[K_X(s)-sx\right]
  \label{eq:saddlePoint:generalSaddlePoint}
\end{equation}
where $K_X(\theta) = \ln\left(M_X(\theta)\right)$, and $s=s(x)$ is the solution to the saddle point equation $K_X'(s)=x$.

For the compound Poisson-gamma distribution, the saddle point approximation is given as 
\begin{multline}
  p_X(x)\approx
  \frac{\left(\lambda\alpha\beta^\alpha\right)^{\frac{1}{2(\alpha+1)}}\euler^{-\lambda}}{\sqrt{2\pi(\alpha+1)}}x^{-\frac{\alpha+2}{2(\alpha+1)}}
  \euler^{-x\beta}
  \exp\left[x^{\frac{\alpha}{\alpha+1}}
    \frac{(\lambda\beta^\alpha)^{\frac{1}{\alpha+1}}(\alpha+1)}{\alpha^{\frac{\alpha}{\alpha+1}}}
  \right]
  \\
  \text{for }x>0 \ .
  \label{eq:saddle_point_approx}
\end{multline}
The approximation is not well defined for $x=0$.

The integral of the density approximation over the support may not equal to one, it can be numerically re-normalised if necessary. Thus it may be more sensible to write the approximation up to a constant
\begin{equation}
  p_X(x)\propto x^{-\frac{\alpha+2}{2(\alpha+1)}}\euler^{-x\beta}
  \exp\left[
    x^{\frac{\alpha}{\alpha+1}}
    \frac{
      (\lambda\beta^\alpha)^{\frac{1}{\alpha+1}}(\alpha+1)
    }
    {
      \alpha^{\frac{\alpha}{\alpha+1}}
    }
  \right]
  \ .
\end{equation}

For the remainder of this section, it will be shown how the saddlepoint approximation can be algebraically derived. First of all the cumulant generating function $K_X(\theta)$ can be obtained from the m.g.f.
\begin{equation}
  K_X(\theta) = \lambda
  \left[
    \left(\frac{\beta}{\beta-\theta}\right)^\alpha-1
  \right]
  \ .
\end{equation}
Taking the derivative with respect to $\theta$
\begin{equation}
  K_X'(\theta)=\frac{\lambda\alpha\beta^\alpha}{(\beta-\theta)^{\alpha+1}}
\end{equation}
and this is known as the saddlepoint equation. The quantity $s=s(x)$ is the solution to the equation $K_X'(s)=x$, that is
\begin{equation*}
  \frac{\lambda\alpha\beta^\alpha}{(\beta-s)^{\alpha+1}} = x
\end{equation*}
with solution
\begin{equation}
  s = \beta - \left(\frac{\lambda\alpha\beta^\alpha}{x}\right)^{\frac{1}{\alpha+1}}
  \ .
\end{equation}

The second derivative of the cumulant generating function is
\begin{equation}
  K_X''(\theta)=\frac{\lambda\alpha(\alpha+1)\beta^\alpha}{(\beta-\theta)^{\alpha+2}} \ .
\end{equation}
Substituting this and $K_X(\theta)$ into Equation \eqref{eq:saddlePoint:generalSaddlePoint}
\begin{equation*}
  p_X(x)\approx
  \frac{1}{\sqrt{2\pi}}
  \left[
    \frac{
      (\beta-s)^{\alpha+2}
    }
    {
      \lambda\alpha(\alpha+1)\beta^\alpha
    }
  \right]^{1/2}
  \exp\left[
    \lambda\left(\left(\frac{\beta}{\beta-s}\right)^\alpha-1\right)-sx
  \right]
  \ .
\end{equation*}
Substituting in $s=s(x)$
\begin{multline*}
  p_X(x)\approx
  \frac{1}{\sqrt{2\pi}}
  \left[
    \frac{\left(\beta-\beta+\left(\frac{\lambda\alpha\beta^\alpha}{x}\right)^{\frac{1}{\alpha+1}}\right)^{\alpha+2}}{\lambda\alpha(\alpha+1)\beta^\alpha}
  \right]^{1/2}
  \\
  \exp\left[
    \lambda
    \left(
      \left(
        \frac{\beta}{\beta-\beta+\left(\frac{\lambda\alpha\beta^\alpha}{x}\right)^{\frac{1}{\alpha+1}}}
      \right)^\alpha
      -1
    \right)
    -x\left(
      \beta-\left(\frac{\lambda\alpha\beta^\alpha}{x}\right)^{\frac{1}{\alpha+1}}
    \right)
  \right]
\end{multline*}
simplifying further
\begin{multline*}
  p_X(x)\approx
  \frac{1}{\sqrt{2\pi}}
  \left[
    \frac{
      \left(\frac{\lambda\alpha\beta^\alpha}{x}\right)^{\frac{\alpha+2}{\alpha+1}}
    }
    {
      \lambda\alpha(\alpha+1)\beta^\alpha
    }
  \right]^{1/2}
  \\
  \exp\left[
    \lambda
    \left(
      \beta^\alpha
      \left(\frac{x}{\lambda\alpha\beta^\alpha}\right)^{\frac{\alpha}{\alpha+1}}
      -1
    \right)
    -x\beta
    +(\lambda\alpha\beta^\alpha)^{\frac{1}{\alpha+1}}x^{1-\frac{1}{\alpha+1}}
  \right]
\end{multline*}
\begin{multline*}
  p_X(x)\approx
  \frac{1}{\sqrt{2\pi(\alpha+1)}}x^{-\frac{\alpha+2}{2(\alpha+1)}}
  \left(\lambda\alpha\beta^\alpha\right)^{\left(\frac{\alpha+2}{\alpha+1}-1\right)/2}
  \\
  \exp\left[
    \lambda\left(
      \beta^\alpha
      \left(\frac{x}{\lambda\alpha\beta^\alpha}\right)^{\frac{\alpha}{\alpha+1}}
      -1
    \right)
    -x\beta
    +(\lambda\alpha\beta^\alpha)^{\frac{1}{\alpha+1}}x^{\frac{\alpha}{\alpha+1}}
  \right]
\end{multline*}
\begin{multline*}
  p_X(x)\approx
  \frac{
    \left(
      \lambda\alpha\beta^\alpha
    \right)^{\frac{1}{2(\alpha+1)}}
  }
  {
    \sqrt{2\pi(\alpha+1)}}x^{-\frac{\alpha+2}{2(\alpha+1)}
  } 
  \\
  \exp\left[
    -\lambda
    -x\beta
    +x^{\frac{\alpha}{\alpha+1}}
    \left(
      \frac{
        \lambda\beta^\alpha
      }
      {
        \left(\lambda\alpha\beta^\alpha\right)^{\frac{\alpha}{\alpha+1}}
      }
      +
      \left(
        \lambda\alpha\beta^\alpha
      \right)^{\frac{1}{\alpha+1}}
    \right)
  \right]
\end{multline*}
\begin{multline*}
  p_X(x)\approx
  \frac{
    \left(
      \lambda\alpha\beta^\alpha
    \right)^{\frac{1}{2(\alpha+1)}}\euler^{-\lambda}
  }
  {
    \sqrt{2\pi(\alpha+1)}}x^{-\frac{\alpha+2}{2(\alpha+1)}
  }
  \euler^{-x\beta}
  \\
  \exp\left[
    x^{\frac{\alpha}{\alpha+1}}
    \left(
      (\lambda\beta^\alpha)^{1-\frac{\alpha}{\alpha+1}}\alpha^{-\frac{\alpha}{\alpha+1}}
      +
      \left(
        \lambda\alpha\beta^\alpha
      \right)^{\frac{1}{\alpha+1}}
    \right)
  \right]
\end{multline*}
\begin{multline*}
  p_X(x)\approx
  \frac{
    \left(\lambda\alpha\beta^\alpha\right)^{\frac{1}{2(\alpha+1)}}\euler^{-\lambda}
  }
  {
    \sqrt{2\pi(\alpha+1)}
  }
  x^{-\frac{\alpha+2}{2(\alpha+1)}}\euler^{-x\beta}
  \\
  \exp\left[
    x^{\frac{\alpha}{\alpha+1}}
    \left(
      (\lambda\beta^\alpha)^{\frac{1}{\alpha+1}}
      \alpha^{-\frac{\alpha}{\alpha+1}}
      +
      \left(\lambda\alpha\beta^\alpha\right)^{\frac{1}{\alpha+1}}
    \right)
  \right]
\end{multline*}
\begin{multline*}
  p_X(x)\approx
  \frac{
    \left(\lambda\alpha\beta^\alpha\right)^{\frac{1}{2(\alpha+1)}}\euler^{-\lambda}
  }
  {
    \sqrt{2\pi(\alpha+1)}
  }
  x^{-\frac{\alpha+2}{2(\alpha+1)}}\euler^{-x\beta}
  \\
  \exp\left[
    x^{\frac{\alpha}{\alpha+1}}(\lambda\beta^\alpha)^{\frac{1}{\alpha+1}}
    \left(
      \alpha^{-\frac{\alpha}{\alpha+1}}+\alpha^{\frac{1}{\alpha+1}}
    \right)
  \right]
  \ .
\end{multline*}

The expression $\alpha^{-\frac{\alpha}{\alpha+1}}+\alpha^{\frac{1}{\alpha+1}}$ can be simplified by putting the two terms over a common denominator
\begin{align}
  \alpha^{-\frac{\alpha}{\alpha+1}}+\alpha^{\frac{1}{\alpha+1}}
  & = \alpha^{\frac{1}{\alpha+1}}+\frac{1}{\alpha^{\frac{\alpha}{\alpha+1}}}
  \nonumber\\
  & = \frac{\alpha^{\frac{1}{\alpha+1}}\alpha^{\frac{\alpha}{\alpha+1}}+1}{\alpha^{\frac{\alpha}{\alpha+1}}}
  \nonumber\\
  & = \frac{\alpha+1}{\alpha^{\frac{\alpha}{\alpha+1}}}
\end{align}
so that
\begin{equation*}
p_X(x)\approx
  \frac{
    \left(\lambda\alpha\beta^\alpha\right)^{\frac{1}{2(\alpha+1)}}\euler^{-\lambda}
  }
  {
    \sqrt{2\pi(\alpha+1)}
  }
  x^{-\frac{\alpha+2}{2(\alpha+1)}}
  \euler^{-x\beta}
  \exp\left[
    x^{\frac{\alpha}{\alpha+1}}
    \frac{(\lambda\beta^\alpha)^{\frac{1}{\alpha+1}}(\alpha+1)}{\alpha^{\frac{\alpha}{\alpha+1}}}
  \right]
  \ .
\end{equation*}

\subsection{Exact Method using Series Evaluation}

The infinite sum can be computationally summed in a clever way. This was done by summing only large terms in the sum and ignores small terms. By using Stirling's approximation, the largest term in the sum can be approximately found \citep{dunn2005series}. The p.d.f.~of the compound Poisson-gamma distribution is difficult because of the infinite sum
\begin{equation}
p_X(x) = 
  \delta(x) \euler^{-\lambda}
  +
  \euler^{-\beta x-\lambda}\frac{1}{x}\sum_{y=1}^{\infty}W_y
  \quad\text{for }x\geqslant 0
\end{equation}
where
\begin{equation}
  W_y = \frac{\beta^{y\alpha}\lambda^yx^{y\alpha}}{\Gamma(y\alpha)y!}
  \ .
  \label{eq:compoundPoisson_Wy}
\end{equation}
$W_y$ can be expressed in a different way using different parameters
\begin{equation}
  W_y(x,p,\phi)=\frac{x^{y\alpha}}{\phi^{y(1+\alpha)}(p-1)^{y\alpha}(2-p)^yy!\Gamma(y\alpha)}
\end{equation}
where
\begin{equation}
  p=\frac{2+\alpha}{1+\alpha}
\end{equation}
and
\begin{equation}
  \phi = \frac{\alpha+1}{\beta^{2-p}(\lambda\alpha)^{p-1}} \ .
\end{equation}

\cite{dunn2005series} approximated the sum by truncation
\begin{equation}
  \sum_{y=1}^\infty W_y \approx \sum_{y=y_\text{l}}^{y_\text{u}}W_y
\end{equation}
where $y_\text{l}<y_{\text{max}}<y_\text{u}$ and $y_{\text{max}}$ is the value of $y$ which maximises $W_y$. \cite{dunn2005series} used Stirling's approximation to find $y_{\text{max}}$. This was done by treating $W_y$ as a continuous function of $y$ and is differentiable with respect to $y$.

It is easier to differentiate $\ln(W_y)$ where
\begin{multline*}
  \ln(W_y) = y\alpha\ln(x)-y(1+\alpha)\ln(\phi)-y\alpha\ln(p-1)\\-y\ln(2-p)-\ln(y!)-\ln\Gamma(y\alpha)
\end{multline*}
\begin{equation}
  \ln(W_y) =
  y
  \ln\left(
    \frac{x^\alpha}{\phi^{1+\alpha}(p-1)^\alpha(2-p)}
  \right)
  -\ln(y!)-\ln\Gamma(y\alpha)
  \ .
\end{equation}
Using Stirling's approximation $\ln(n!)\approx\ln\Gamma(n)\approx n\ln(n!)-n$, then
\begin{equation}
  \ln(W_y) \approx
  y\ln\left(
    \frac{x^\alpha}{\phi^{1+\alpha}(p-1)^\alpha(2-p)}
  \right)
  -y\ln(y)+y-y\alpha\ln(y\alpha) + y\alpha
  \ .
\end{equation}
Taking the derivative with respect to $y$
\begin{multline*}
  \frac{\partial \ln(W_y)}{\partial y} \approx
  \ln\left(
    \frac{x^\alpha}{\phi^{1+\alpha}(p-1)^\alpha(2-p)}
  \right)
  -\ln(y)-1+1
  \\
  -\alpha\ln(y\alpha)-\alpha+\alpha
\end{multline*}
\begin{equation}
  \frac{\partial \ln(W_y)}{\partial y} \approx
  \ln\left(
    \frac{x^\alpha}{\phi^{1+\alpha}(p-1)^\alpha(2-p)}
  \right)
  -\ln(y)
  -\alpha\ln(y\alpha)
  \ .
\end{equation}

Setting the derivative to zero
\begin{equation*}
  0 \approx \ln\left(
    \frac{
      x^\alpha
    }
    {
      \phi^{1+\alpha}(p-1)^\alpha(2-p)y_{\text{max}}^{1+\alpha}\alpha^\alpha
    }
  \right)
\end{equation*}
\begin{equation*}
  1 \approx 
  \frac{x^\alpha}{\phi^{1+\alpha}(p-1)^\alpha(2-p)y_{\text{max}}^{1+\alpha}\alpha^\alpha}
\end{equation*}
\begin{equation*}
  y_{\text{max}}^{1+\alpha} \approx 
  \frac{x^\alpha}{\phi^{1+\alpha}(p-1)^\alpha(2-p)\alpha^\alpha}
\end{equation*}
\begin{equation*}
  y_{\text{max}} \approx 
  \frac{1}{\phi}
  \left(
    \frac{x}{(p-1)\alpha}
  \right)^{\frac{\alpha}{1+\alpha}}
  (2-p)^{\frac{-1}{1+\alpha}}
  \ .
\end{equation*}
This can be simplified using the fact that 
\begin{equation*}
  \alpha=\frac{2-p}{p-1}
  \ ,
\end{equation*}
\begin{equation*}
  \frac{1}{1+\alpha} = p-1
\end{equation*}
and
\begin{equation*}
  \frac{\alpha}{1+\alpha} = 2-p
\end{equation*}
then
\begin{equation*}
  y_{\text{max}} \approx 
  \frac{1}{\phi}
  \left(
    \frac{x}{2-p}
  \right)^{2-p}
  (2-p)^{1-p}
\end{equation*}
and finally
\begin{equation}
  y_{\text{max}} \approx \frac{x^{2-p}}{\phi(2-p)}
  \ .
\end{equation}
Because values of $y$ should be a positive integer, it would be appropriate to round the solution to $y_\text{max}$ accordingly
\begin{equation}
  y_{\text{max}} = \text{max}\left[
    1,\text{round}\left(\frac{x^{2-p}}{\phi(2-p)}\right)
  \right]
  \ .
\end{equation}

To verify that $y_\text{max}$ is a maximum, the second derivative can be investigated
\begin{equation}
  \frac{\partial^2\ln(W_y)}{\partial y^2}
  \approx
  -\frac{1}{y}(\alpha+1)
\end{equation}
to see that
\begin{equation}
  \frac{\partial^2\ln(W_y)}{\partial y^2} < 0 \quad \text{for }y=1,2,3,\dotdotdot
\end{equation}
therefore $y_\text{max}$ is a maximum.

Returning back to the truncation of the infinite sum
\begin{equation}
  \sum_{y=1}^\infty W_y \approx \sum_{y=y_\text{l}}^{y_\text{u}}W_y
  \ ,
\end{equation}
$y_\text{l}$ and $y_\text{u}$ can be chosen such that $W_{y_\text{l}}$ and $W_{y_\text{u}}$ are less than $\epsilon W_{y_\text{max}}$ where $\epsilon$ is some small constant, for example $\epsilon=\euler^{-37}$ will be better than machine precision in 64 bits \citep{dunn2005series}. To prevent overflow problems, it is advised to calculate each term in the summation in log scale \citep{dunn2005series} and using the following equation
\begin{equation}
  \ln\left[
    \sum_{y=y_\text{l}}^{y_\text{u}}W_y
  \right]
  = 
  \ln\left(
    W_{y_\text{max}}
  \right)
  +\ln\sum_{y=y_\text{l}}^{y_\text{u}}
  \exp\left[
    \ln\left(W_y\right)-\ln\left(W_{y_\text{max}}\right)
  \right]
  \ .
\end{equation}

\section{Simulation Studies on Density Evaluation}

Simulations of a compound Poisson-gamma random variable were conducted with the aim to compare how well these density evaluations methods perform. This was done by comparing the evaluated densities with the histogram of simulations and a Q-Q plot. The following compound Poisson-gamma distributions were used in the simulations to capture the variety in the compound Poisson-gamma family: $\CPoisson(1,1,1)$, $\CPoisson(10,1,1)$, $\CPoisson(1,100,1)$, $\CPoisson(100,100,1)$.

A few technical problems with the histogram because the compound Poisson has probability mass at zero and probability density for positive numbers. To correctly represent the empirical distribution of a compound Poisson random variable, a bar chart was used to show the frequency of a zero and a histogram to show the frequency density of positive numbers. However, the Normal approximation and the saddlepoint approximation does not have support at zero. Therefore to compare these approximate densities to the empirical distribution fairly, a histogram containing both zero and positive samples was used when appropriate.

The evaluation of the p.d.f.~using the saddlepoint approximate required a bit of caution to avoid over/underflow problems. Suppose realizations of $X$ were simulated $\left\{x_1,x_2,x_3,\dotdotdot,x_n\right\}$. The saddlepoint approximation was computed up to a constant using
\begin{equation}
  p_X(x) \propto
  \exp\left[
    -\frac{\alpha+2}{2(\alpha+1)}
    \ln(x)
    -x\beta
    +\left(
    \frac{x\beta}{\alpha}
    \right)^{\frac{\alpha}{\alpha+1}}\lambda^{\frac{1}{\alpha+1}}(\alpha+1) - k
  \right]
\end{equation}
for 500 equally spaced points from and including the minimum non-zero simulated value to the maximum simulated value. $k$ was chosen to be
\begin{equation}
  k =
  \max_{i\in\left\{1,2,3,\dotdotdot,n\right\}}
  \left[
    -\frac{\alpha+2}{2(\alpha+1)}
    \ln(x_i)
    -x_i\beta
    +\left(
      \frac{x_i\beta}{\alpha}
    \right)^{\frac{\alpha}{\alpha+1}}\lambda^{\frac{1}{\alpha+1}}(\alpha+1)
  \right]
  \ .
\end{equation}
The density was then was normalised by numerically integrating it using the trapezium rule using the 500 evaluated points.

A Q-Q plot is a plot which compares the empirical c.d.f.~with the theoretical c.d.f. Let
\begin{equation}
  F_X(x) = \prob(X\leqslant x)
\end{equation}
and
\begin{equation}
  \widehat{F}_X(x) = \frac{1}{n}\cdot\text{max}
  \left[
    \left(\sum_{i=1}^n\mathbb{I}(x_i\leqslant x)\right)-0.5,0
  \right]
  \ ,
\end{equation}
then a Q-Q plot is a parametric plot which plots $\widehat{F}_X^{-1}(p)$ against $F_X^{-1}(p)$ for $p=\frac{0.5}{n},\frac{1.5}{n},\frac{2.5}{n},\dotdotdot,\frac{n-0.5}{n}$. If $F_X(x)$ and $\widehat{F}_X(x)$ are similar, a Q-Q plot should be a straight line with gradient 1, intercepting the origin. For the exact method and the saddlepoint approximate, $F_X(x)$ was found numerically by evaluating the p.d.f.~at $10\,000$ equally spaced points from and including the minimum to the maximum of the simulated samples, and then summing the required trapeziums. $\widehat{F}_X^{-1}(p)$ was then calculated by interpolation. Because the Normal and saddlepoint approximation does not support zero, the Q-Q plot was omitted for these approximations for large $\lambda$.

When plotting the probability of obtaining a zero or the p.d.f.~for positive values, confidence intervals were plotted as well. Consider a bin in a histogram, the confidence intervals were obtained by assuming that the frequency in a bin $\sim\poisson(\text{p.d.f.~evaluated at the bin}\times 1\,000 \times \text{bin width})$.

For low $\lambda$ (Figures \ref{fig:compoundPoisson_histogram_1_1_1}, \ref{fig:compoundPoisson_histogram_1_100_1}, \ref{fig:compoundPoisson_histogram_approx_1_1_1} and \ref{fig:compoundPoisson_histogram_approx_1_100_1}) there is a chance of simulating zeros. For $\lambda = 1$, the Normal approximation failed to capture the probability mass at zero because the Normal distribution is symmetric and support negative values. The saddlepoint approximation did capture the probability mass at zero as shown by an increase in probability density towards zero. For $\CPoisson(1,100,1)$ in Figure \ref{fig:compoundPoisson_histogram_approx_1_100_1}, the compound Poisson-gamma p.d.f.~contains multiple peaks and the saddlepoint approximation was not flexible enough them. In Figure \ref{fig:compoundPoisson_histogram_1_100_1}, the Q-Q plot for the exact method was quite sensitive at the tails of each peak, perhaps there were a few inaccuracies in the exact evaluation of the p.d.f.~or in the calculation or inversion of the c.d.f.

As $\lambda$ increased, in Figures \ref{fig:compoundPoisson_histogram_10_1_1} and \ref{fig:compoundPoisson_histogram_100_1_1}, all 3 density evaluation methods performed quite well. For $\CPoisson(10,1,1)$ in Figure \ref{fig:compoundPoisson_histogram_10_1_1}, the Normal approximation did not quite capture the skewness. In Figure \ref{fig:compoundPoisson_histogram_100_1_1}, $\lambda$ was high enough where the compound-Poisson distribution starts to converge to a Normal distribution. All methods performed well in this realm.

\begin{figure}
  \centering
    \centerline{
    \begin{subfigure}[b]{\mainSize}
        \includegraphics[width=\textwidth]{../figures/compoundpoisson/cpHistogram_CompoundPoissonlambda1alpha1beta1.eps}
        \caption{Left: Expected and observed frequency of a zero. Right: Histogram and p.d.f.~of non-zero values.}
    \end{subfigure}
    }
    \vspace{2em}
    \centerline{
    \begin{subfigure}[b]{\subSize}
        \includegraphics[width=\textwidth]{../figures/compoundpoisson/cpHistogram_qq_CompoundPoissonlambda1alpha1beta1.eps}
        \caption{Q-Q plot}
    \end{subfigure}
    }
    \caption{1\,000 simulations of a $\CPoisson(1,1,1)$ random variable were simulated and its empirical distribution is compared to the p.d.f.~evaluated using the exact method. In a), the dotted red line shows the $\Phi(\pm1)$ quantile of the expected frequency or frequency density.}
    \label{fig:compoundPoisson_histogram_1_1_1}
\end{figure}

\begin{figure}
  \centering
    \centerline{
    \begin{subfigure}[b]{\mainSize}
        \includegraphics[width=\textwidth]{../figures/compoundpoisson/cpHistogram_CompoundPoissonlambda1alpha100beta1.eps}
        \caption{Left: Expected and observed frequency of a zero. Right: Histogram and p.d.f.~of non-zero values.}
    \end{subfigure}
    }
    \vspace{2em}
    \centerline{
    \begin{subfigure}[b]{\subSize}
        \includegraphics[width=\textwidth]{../figures/compoundpoisson/cpHistogram_qq_CompoundPoissonlambda1alpha100beta1.eps}
        \caption{Q-Q plot}
    \end{subfigure}
    }
    \caption{1\,000 simulations of a $\CPoisson(1,100,1)$ random variable were simulated and its empirical distribution is compared to the p.d.f.~evaluated using the exact method. In a), the dotted red line shows the $\Phi(\pm1)$ quantile of the expected frequency or frequency density.}
    \label{fig:compoundPoisson_histogram_1_100_1}
\end{figure}

\begin{figure}
  \centering
    \centerline{
    \begin{subfigure}[b]{\subSize}
        \includegraphics[width=\textwidth]{../figures/compoundpoisson/cpHistogram_CompoundPoissonNormlambda1alpha1beta1.eps}
        \caption{Normal approximation}
    \end{subfigure}
    \begin{subfigure}[b]{\subSize}
        \includegraphics[width=\textwidth]{../figures/compoundpoisson/cpHistogram_CompoundPoissonSaddlelambda1alpha1beta1.eps}
        \caption{Saddlepoint approximation}
    \end{subfigure}
    }
    \caption{1\,000 simulations of a $\CPoisson(1,1,1)$ random variable were simulated and its empirical distribution is compared to the p.d.f.~evaluated using approximate methods. In a), the dotted red line shows the $\Phi(\pm1)$ quantile of the expected frequency density.}
    \label{fig:compoundPoisson_histogram_approx_1_1_1}
\end{figure}

\begin{figure}
  \centering
    \centerline{
    \begin{subfigure}[b]{\subSize}
        \includegraphics[width=\textwidth]{../figures/compoundpoisson/cpHistogram_CompoundPoissonNormlambda1alpha100beta1.eps}
        \caption{Normal approximation}
    \end{subfigure}
    \begin{subfigure}[b]{\subSize}
        \includegraphics[width=\textwidth]{../figures/compoundpoisson/cpHistogram_CompoundPoissonSaddlelambda1alpha100beta1.eps}
        \caption{Saddlepoint approximation}
    \end{subfigure}
    }
    \caption{1\,000 simulations of a $\CPoisson(1,100,1)$ random variable were simulated and its empirical distribution is compared to the p.d.f.~evaluated using approximate methods. In a), the dotted red line shows the $\Phi(\pm1)$ quantile of the expected frequency density.}
    \label{fig:compoundPoisson_histogram_approx_1_100_1}
\end{figure}

\begin{figure}
  \centering
    \centerline{
    \begin{subfigure}[b]{\subSize}
        \includegraphics[width=\textwidth]{../figures/compoundpoisson/cpHistogram_CompoundPoissonlambda10alpha1beta1.eps}
        \caption{Histogram - Exact method}
    \end{subfigure}
    \begin{subfigure}[b]{\subSize}
        \includegraphics[width=\textwidth]{../figures/compoundpoisson/cpHistogram_qq_CompoundPoissonlambda10alpha1beta1.eps}
        \caption{Q-Q plot - Exact method}
    \end{subfigure}
    }
    \centerline{
    \begin{subfigure}[b]{\subSize}
        \includegraphics[width=\textwidth]{../figures/compoundpoisson/cpHistogram_CompoundPoissonNormlambda10alpha1beta1.eps}
        \caption{Histogram - Normal approx.}
    \end{subfigure}
    \begin{subfigure}[b]{\subSize}
        \includegraphics[width=\textwidth]{../figures/compoundpoisson/cpHistogram_qq_CompoundPoissonNormlambda10alpha1beta1.eps}
        \caption{Q-Q plot - Normal approx.}
    \end{subfigure}
    }
    \centerline{
    \begin{subfigure}[b]{\subSize}
        \includegraphics[width=\textwidth]{../figures/compoundpoisson/cpHistogram_CompoundPoissonSaddlelambda10alpha1beta1.eps}
        \caption{Histogram - Saddlepoint approx.}
    \end{subfigure}
    \begin{subfigure}[b]{\subSize}
        \includegraphics[width=\textwidth]{../figures/compoundpoisson/cpHistogram_qq_CompoundPoissonSaddlelambda10alpha1beta1.eps}
        \caption{Q-Q plot - Saddlepoint approx.}
    \end{subfigure}
    }
    \caption{1\,000 simulations of a $\CPoisson(10,1,1)$ random variable were simulated and its empirical distribution is compared to the p.d.f. On the left, the dotted red line shows the $\Phi(\pm1)$ quantile of the expected frequency density.}
    \label{fig:compoundPoisson_histogram_10_1_1}
\end{figure}

\begin{figure}
  \centering
    \centerline{
    \begin{subfigure}[b]{\subSize}
        \includegraphics[width=\textwidth]{../figures/compoundpoisson/cpHistogram_CompoundPoissonlambda100alpha100beta1.eps}
        \caption{Histogram - Exact method}
    \end{subfigure}
    \begin{subfigure}[b]{\subSize}
        \includegraphics[width=\textwidth]{../figures/compoundpoisson/cpHistogram_qq_CompoundPoissonlambda100alpha100beta1.eps}
        \caption{Q-Q plot - Exact method}
    \end{subfigure}
    }
    \centerline{
    \begin{subfigure}[b]{\subSize}
        \includegraphics[width=\textwidth]{../figures/compoundpoisson/cpHistogram_CompoundPoissonNormlambda100alpha100beta1.eps}
        \caption{Histogram - Normal approx.}
    \end{subfigure}
    \begin{subfigure}[b]{\subSize}
        \includegraphics[width=\textwidth]{../figures/compoundpoisson/cpHistogram_qq_CompoundPoissonNormlambda100alpha100beta1.eps}
        \caption{Q-Q plot - Normal approx.}
    \end{subfigure}
    }
    \centerline{
    \begin{subfigure}[b]{\subSize}
        \includegraphics[width=\textwidth]{../figures/compoundpoisson/cpHistogram_CompoundPoissonSaddlelambda100alpha100beta1.eps}
        \caption{Histogram - Saddlepoint approx.}
    \end{subfigure}
    \begin{subfigure}[b]{\subSize}
        \includegraphics[width=\textwidth]{../figures/compoundpoisson/cpHistogram_qq_CompoundPoissonSaddlelambda100alpha100beta1.eps}
        \caption{Q-Q plot - Saddlepoint approx.}
    \end{subfigure}
    }
    \caption{1\,000 simulations of a $\CPoisson(100,100,1)$ random variable were simulated and its empirical distribution is compared to the p.d.f. On the left, the dotted red line shows the $\Phi(\pm1)$ quantile of the expected frequency density.}
    \label{fig:compoundPoisson_histogram_100_1_1}
\end{figure}

\section{Proposed Model}

The compound Poisson-gamma distribution can be used to model the grey values of each pixel in a projection. Suppose a projection has $m$ pixels and $n$ replicate projections were obtained. Let $X_{i,j}$ and $Y_{i,j}$ be the grey value and photon count, respectively, of the $i$th pixel in the $j$th replicate projection.

By assuming no beam hardening, the distribution of the photon does not change with attenuation. As a result all pixels will detect photons with identical energy distributions thus $\alpha$ and $\beta$ are the same for all pixels. Attenuation does affect the photon count, the more material a photon has to attenuate, the lower the number of detectable photons. The amount of attenuation depends on the specific path from the source to a pixel in a detector, so $\lambda_i$ varies from pixel to pixel. Electronic Normal noise $\epsilon_{i,j}\sim\normal(a,\kappa)$ may be added as well.

\begin{figure}
  \centering
  \includegraphics[width=0.6\textwidth]{../figures/compoundpoisson/graphicalModel.eps}
  \caption{Graphical model of the grey value $X_{i,j}$ for each of the $m$ pixels in the $n$ replicate projections. $Y_{i,j}\sim\poisson(\lambda_i)$ is the photon count. The grey value has a compound Poisson gamma element $X_{i,j}|Y_{i,j}\sim\gammaDist(Y_{i,j}\alpha,\beta)$ and can be extended by adding electronic noise $\epsilon_{i,j}\sim\normal(a,\kappa)$.}
  \label{fig:compoundPoisson_graphicalModel}
\end{figure}

The graphical model in Figure \ref{fig:compoundPoisson_graphicalModel} illustrates how all of these variables link together. $a$ and $\kappa$ are unknown parameters but can be estimated beforehand using replicate black images. The energy of each photon $U\sim\gammaDist(\alpha,\beta)$ was omitted because the conditional distribution $X|Y\sim\gammaDist(Y\alpha,\beta)$ encapsulates each detected photon energy already.

The model may be simplified by omitting the electronic noise, this can be done by setting $a=0$ and $\kappa=0$. The number of pixels can be set to $m=1$ as well. The model is well set up for the EM algorithm \citep{dempster1977maximum}. If each $Y_{i,j}$ is known, then $\lambda_i$, $\alpha$ and $\beta$ can be estimated using maximum likelihood. $Y_{i,j}$ could be estimated if $\lambda_i$, $\alpha$ and $\beta$ are known. This problem can be tackled using the EM algorithm.

\section{EM Algorithm}

The EM algorithm \citep{dempster1977maximum} will be proposed here as a method to estimate the parameters of a $X\sim\CPoisson(\lambda,\alpha,\beta)$ random variable given a sample of measurements of it $\left\{x_1,x_2,x_3,\dotdotdot,x_n\right\}$.

A popular method for parameter estimation is the maximum log likelihood method. Let $\widehat{\lambda}$, $\widehat{\alpha}$ and $\widehat{\beta}$ be estimators of $\lambda$, $\alpha$ and $\beta$ respectively. The log likelihood is defined to be
\begin{equation*}
	\ln L(\lambda,\alpha,\beta;X) = \sum_{i=1}^n
	\left[
		\mathbb{I}(x_i=0)\ln \prob(X=x_i)
		+\mathbb{I}(x_i>0)\ln p_X(x_i)
	\right]
\end{equation*}
\begin{multline}
	\ln L(\lambda,\alpha,\beta;X) = \sum_{i=1}^n
	\left[
		-\mathbb{I}(x_i=0)\lambda
		\vphantom{\sum_i^i}
	\right.
	\\
	\left.
		+\mathbb{I}(x_i>0)\left(
		-1-\beta x_i-\lambda-\ln x -\ln\sum_{y=1}^\infty W_y
		\right)
	\right]
\end{multline}
where $W_y=W_y(\lambda,\alpha,\beta)$ was defined in Equation \eqref{eq:compoundPoisson_w}. $\widehat{\lambda}$, $\widehat{\alpha}$ and $\widehat{\beta}$ are values of $\lambda$, $\alpha$ and $\beta$ which jointly maximises the log likelihood. The gradient of the log likelihood can be evaluated \citep{dunn2005series} and optimisation can be done using gradient methods.

The EM algorithm \citep{dempster1977maximum} instead treats $Y\sim\poisson(\lambda)$ as a latent variable. Let $\left\{Y_1,Y_2,Y_3,\dotdotdot, Y_n\right\}$ be realizations of $Y$ and $\left\{x_1,x_2,x_3,\dotdotdot, x_n\right\}$ be measurements of $X|Y$. Define the joint log likelihood to be
\begin{multline*}
	\ln L(\lambda,\alpha,\beta;X,Y)
	=
	\sum_{i=1}^n
	\left[
		\mathbb{I}(x_i=0)
		\ln
		\prob(Y=0)
	\right.
	\\
	\left.
		+
		\mathbb{I}(x_i>0)
		\ln
		\left[
			p_{X|Y}(x_i|Y_i)\prob(Y=Y_i)
		\right]
	\right]
\end{multline*}
so that
\begin{multline}
	\ln L(\lambda,\alpha,\beta;X,Y)=
	\sum_{i=1}^n
	\left[
		-\mathbb{I}(x_i=0)
		\lambda
	\right.
	\\
	\left.+
		\mathbb{I}(x_i>0)
		\left(
			Y_i\alpha\ln\beta-\ln\Gamma(Y_i\alpha)+(Y_i\alpha-1)\ln x_i - \beta x_i
		\right.
	\right.
	\\
	\left.
		\left.	
			- \lambda + Y_i \ln \lambda - \ln(Y_i!)
		\right)
	\right]
	\ .
\end{multline}
The estimators are found by optimising the joint log likelihood. This is done iteratively by estimating the $Y$s given the $X$s and parameters (E step), followed by estimating the parameters given the $Y$s and $X$s (M step) until some convergence conditions are met.

It will be shown that using some approximations, the E step and M step can be implemented. The Cram\'er-Rao lower bound \citep{rao1945information} \citep{cramer1946mathematical} can also be found by using a few approximations. However simulations show that these estimators struggle for high $\lambda$ and it will be shown that this will be the case for all estimators.

\subsection{E Step}

In the E step, the realizations of $Y$ are estimate using
\begin{equation}
	y_i =
	\expectation\left[
		Y|X=x_i
	\right]
\end{equation}
for given parameters $\lambda$, $\alpha$ and $\beta$. The conditional expectation is calculated using
\begin{equation}
	y_i = 
	\begin{cases}
		0 & \text{ for } x=0 \\ 
		\dfrac{\sum_{y=1}^\infty y \prob(Y=y|X=x_i)}{\sum_{y=1}^\infty \prob(Y=y|X=x_i)} & \text{ for } x>0
	\end{cases}
\end{equation}
where
\begin{equation*}
	\prob(Y=y|X=x) = \frac{p_{X|Y}(x|y)\prob(Y=y)}{p_X(x)}
	\ .
\end{equation*}
Focussing on the $x>0$ case for now
\begin{equation*}
	\prob(Y=y|X=x) = \frac{1}{p_X(x)}\frac{\beta^{y\alpha}}{\Gamma(y\alpha)}x^{y\alpha-1}\euler^{-\beta x}\frac{\euler^{-\lambda}\lambda^y}{y!}
\end{equation*}
and
\begin{equation}
	\prob(Y=y|X=x) = W_y \frac{\euler^{-\lambda-\beta x}}{x}
\end{equation}
where $W_y$ is defined in Equation \eqref{eq:compoundPoisson_Wy}.
Then the conditional expectation is
\begin{equation}
	y_i = \frac{\sum_{y=1}^\infty y W_y}{\sum_{y=1}^\infty W_y}
	\ .
\end{equation}
It was already established how the denominator can be evaluated \citep{dunn2005series}. A similar method for evaluating the numerator can be obtained by truncating the sum and summing large terms.

Let
\begin{equation}
	W_y^{(r)} = y^r W_y
\end{equation}
so that
\begin{equation}
	y_i = \frac{\sum_{y=1}^\infty W^{(1)}_y}{\sum_{y=1}^\infty W_y}
\end{equation}
and
\begin{equation}
	\variance[Y|X=x_i] = \frac{\sum_{y=1}^\infty W^{(2)}_y}{\sum_{y=1}^\infty W_y} - \left(y_i\right)^2
	\ .
\end{equation}
The exponent $r=0,1,2,\dotdotdot$ will be useful in the M step where higher conditional moments are needed.
The sum can be truncated
\begin{equation}
	\sum_{y=1}^\infty W^{(r)}_y \approx \sum_{y=y_\text{l}}^{y_\text{u}} W^{(r)}_y
\end{equation}
where $y_\text{l}<y_\text{max}<y_\text{u}$ and $y_\text{max}$ is the value of $y$ which maximises $W_y^{(r)}$. By expressing $\ln(W^{(r)}_y)$ as
\begin{equation}
	\ln\left(W_y^{(r)}\right)=r\ln(y)+\ln(W_y)
\end{equation}
and then taking the derivative with respect to $y$
\begin{equation}
	\frac{\partial \ln(W_y^{(r)})}{\partial y} = \frac{r}{y } + \frac{\partial \ln(W_y)}{\partial y}
	\ .
\end{equation}
Keep in mind that $y=1,2,3,\dotdotdot$ so for large $y$, an approximation can be made $r/y\approx 0$ so that
\begin{equation}
	\frac{\partial \ln(W_y^{(r)})}{\partial y} \approx \frac{\partial \ln(W_y)}{\partial y}
	\ .
\end{equation}
By using such an approximation, the method for evaluating the sum $\sum_{y=1}^\infty W^{(r)}_y$ is almost exactly the same as evaluating the sum $\sum_{y=1}^\infty W_y$, in particular $y_\text{max}$ are the same. For a given $r=0,1,2,\dotdotdot$, the limit of the sum $\sum_{y=y_\text{l}}^{y_\text{u}} W^{(r)}_y$ are chosen such that $W_{y_\text{l}}$ and $W_{y_\text{u}}$ are less than $\epsilon W_{y_\text{max}}^{(r)}$ where $\epsilon$ is some small constant. The limits will be different for different values of $r$.

\subsection{M Step}

In the M step, the conditional expected joint log likelihood is maximised with respect to the parameters $\lambda$, $\alpha$ and $\beta$. The objective function is then
\begin{multline*}
	T(\lambda,\alpha,\beta)=
	\sum_{i=1}^n
	\expectation\left[
		-\mathbb{I}(x_i=0)
		\lambda
	\right.
	\\
	\left.+
		\mathbb{I}(x_i>0)
		\left(
			Y_i\alpha\ln\beta-\ln\Gamma(Y_i\alpha)+(Y_i\alpha-1)\ln x_i - \beta x_i
		\right.
	\right.
	\\
	\left.
		\left.	
			- \lambda + Y_i \ln \lambda - \ln(Y_i!)
		\right)
		|X_i=x_i
	\right]
\end{multline*}
\begin{multline*}
	T(\lambda,\alpha,\beta)=
	-n\lambda
	\\
	+\sum_{i=1}^n
	\mathbb{I}(x_i>0)
	\left[
		\expectation[Y_i|X_i=x_i]\alpha\ln\beta-\expectation[\ln\Gamma(Y_i\alpha)|X_i=x_i]
	\right.
	\\
	\left.
		+\expectation[Y_i|X_i=x_i]\alpha\ln x_i - \beta x_i
		+ \expectation[Y_i|X_i=x_i] \ln \lambda
	\right] + c
\end{multline*}
where $c$ is some constant not dependent on $\lambda$, $\alpha$ or $\beta$.

The conditional expectation $y_i = \expectation[Y_i|X_i=x_i]$ was calculated in the E step. The quantity $\expectation[\ln\Gamma(Y_i\alpha)|X_i=x_i]$ can be calculated using the approximation
\begin{equation}
	\expectation[\ln\Gamma(Y_i\alpha)|X_i=x_i] \approx
	\ln\Gamma(\alpha y_i) + \frac{1}{2}\zeta_i\alpha^2\psi'(y_i\alpha)
\end{equation}
where
\begin{equation}
	\zeta_i = \variance[Y_i|X_i=x_i]
\end{equation}
calculated in the E step and $\psi$ is the digamma function. The objective function is then
\begin{multline}
	T(\lambda,\alpha,\beta)=
	-n\lambda
	+\sum_{i=1}^n
	\mathbb{I}(x_i>0)
	\left[
		y_i\alpha\ln\beta-\ln\Gamma(\alpha y_i) - \frac{1}{2}\zeta_i\alpha^2\psi'(y_i\alpha)
	\right.
	\\
	\left.
		+y_i\alpha\ln x_i - \beta x_i
		+ y_i \ln \lambda
		\vphantom{\frac{1}{1}}
	\right]
	+ c
	\ .
\end{multline}

Taking the derivative with respect to $\lambda$
\begin{equation}
	\frac{\partial T}{\partial\lambda} = -n + \frac{\sum_{i=1}^ny_i}{\lambda}
\end{equation}
and setting it to zero
\begin{equation}
	\widehat{\lambda} = \frac{\sum_{i=1}^n y_i}{n}
\end{equation}
obtains a M step estimator, and a familiar one, for $\lambda$. Taking the second derivative with respect to $\lambda$
\begin{equation}
	\frac{\partial^2 T}{\partial \lambda^2} = -\frac{\sum_{i=1}^ny_i}{\lambda^2} < 0
\end{equation}
verifies $\widehat{\lambda}$ maximises T. In addition
\begin{equation}
	\frac{\partial^2 T}{\partial \alpha \partial \lambda } = 0
\end{equation}
and
\begin{equation}
	\frac{\partial^2 T}{\partial \beta \partial \lambda } = 0
	\ .
\end{equation}

Maximising $T$ with respect to $\alpha$ and $\beta$ can be done numerically using the Newton-Raphson method since derivatives up to the second order can be obtained. For the first order these are:
\begin{multline}
	\frac{\partial T}{\partial \alpha} =
	\sum_{i=1}^n\mathbb{I}(x_i>0)
	\left[
		y_i\ln\beta -\psi(\alpha y_i)y_i-\zeta_i\alpha\psi'(\alpha y_i)
		\vphantom{\frac{1}{1}}
	\right.
	\\
	\left.
		- \frac{1}{2}\zeta_i\alpha^2\psi''(\alpha y_i)y_i + y_i\ln x_i
	\right]
\end{multline}
and
\begin{equation}
	\frac{\partial T}{\partial \beta} = \sum_{i=1}^n\mathbb{I}(x_i>0)\left[
	\frac{\alpha y_i}{\beta}-x_i
	\right]
	\ .
\end{equation}
The second orders are:
\begin{equation}
	\frac{\partial^2 T}{\partial \alpha \partial \beta} =
	\sum_{i=1}^n \mathbb{I}(x_i>0)\left[\frac{y_i}{\beta}\right]
	\ ,
	\label{eq:compoundPoisson:d2tdadb}
\end{equation}
\begin{equation}
	\frac{\partial^2 T}{\partial \beta^2} = \sum_{i=1}^n\mathbb{I}(x_i>0)\left[-\frac{\alpha y_i}{\beta^2}
	\right]
	\label{eq:compoundPoisson:d2td2b}
\end{equation}
and
\begin{multline*}
	\frac{\partial^2 T}{\partial \alpha^2} = 
	\sum_{i=1}^n\mathbb{I}(x_i>0)
	\left[
		-y_i^2\psi'(\alpha y_i) - \zeta_i\psi'(\alpha y_i) - \zeta_i\alpha y_i\psi''(\alpha y_i)
		\vphantom{\frac{1}{2}}\right.\\\left.	
		- \zeta_i\alpha\psi''(\alpha y_i)y_i
		-\frac{1}{2}\zeta_i\alpha^2\psi'''(\alpha y_i)y_i^2
	\right]
\end{multline*}
simplifying to
\begin{multline}
	\frac{\partial^2 T}{\partial \alpha^2} = 
	\sum_{i=1}^n\mathbb{I}(x_i>0)
	\left[
		-(y_i^2+\zeta_i)\psi'(\alpha y_i) - 2\zeta_i\alpha y_i\psi''(\alpha y_i)
		\vphantom{\frac{1}{2}}
	\right.
	\\
	\left.	
		-\frac{1}{2}\zeta_i\alpha^2y_i^2\psi'''(\alpha y_i)
	\right] \ .
	\label{eq:compoundPoisson:d2td2a}
\end{multline}

All the derivatives can be used in the Newton-Raphson iterative update, which is
\begin{equation}
	\begin{pmatrix}
		\alpha \\ \beta
	\end{pmatrix}
	\leftarrow
	\begin{pmatrix}
		\alpha \\ \beta
	\end{pmatrix}
	\left[
		\nabla_{\alpha,\beta}\nabla_{\alpha,\beta}\T T
	\right]^{-1}
	\left[
		\nabla_{\alpha,\beta} T
	\right]
\end{equation}
where
\begin{equation}
	\nabla_{\alpha,\beta}=
	\begin{pmatrix}
		{\partial}/{\partial \alpha}
		\\
		{\partial}/{\partial \beta}
	\end{pmatrix}
	\ .
\end{equation}
Since increasing $T$ is sufficient for the EM algorithm \citep{dempster1977maximum}, one step of the Newton-Raphson iterative update was chosen in the M step to avoid implementing a convergence condition.

\subsection{Estimation Error}

The variance of the M step estimators can be obtained by using the Cram\'er-Rao lower bound \citep{rao1945information} \citep{cramer1946mathematical} from the Fisher's information matrix defined as
\begin{equation}
	\matr{I} = -\expectation\left[
		\nabla_{\lambda,\alpha,\beta}\nabla_{\lambda,\alpha,\beta}\T \ln L(\lambda,\alpha,\beta;X,Y)
	\right]
	\ .
\end{equation}
The calculation is similar to the derivatives of $T$, so using the same approximations and the fact that $\expectation[Y]=\lambda$, then
\begin{equation}
	\matr{I}=
	\begin{pmatrix}
		\dfrac{n}{\lambda} & 0 & 0 \\
		0 & (\lambda^2+\lambda)\psi'(\alpha\lambda)+2\alpha\lambda^2\psi''(\alpha\lambda)+\frac{1}{2}\alpha^2\lambda^3\psi'''(\alpha\lambda) & -\dfrac{n\lambda}{\beta}\\
		0 & -\dfrac{n\lambda}{\beta} & \dfrac{n\alpha\lambda}{\beta^2}
	\end{pmatrix}
	\ .
\end{equation}
The Cram\'er-Rao lower bound is then
\begin{equation}
	\cov\left[
		\begin{pmatrix}
			\widehat{\lambda}\\\widehat{\alpha}\\\widehat{\beta}
		\end{pmatrix}
	\right]
	=
	\matr{I}^{-1}
	\ .
\end{equation}

\subsection{Simulations}

An experiment was conducted to assess the performance on the EM algorithm. For a given set of parameters, 1\,000 samples of a $\CPoisson(\lambda,\alpha,\beta)$ random variable were simulated. The EM algorithm was initialised with its parameters at the true values to investigate the convergence in that vicinity. The log likelihood $\ln L(\lambda,\alpha,\beta;X)$ and the parameters were recorded at every EM step. The experiment was repeated 10 times using different simulated samples.

The results for $\CPoisson(1,1,1)$, $\CPoisson(10,1,1)$, $\CPoisson(1,100,1)$ and $\CPoisson(100,100,1)$ are shown in Figures \ref{fig:compoundPoisson_convergence_1}, \ref{fig:compoundPoisson_convergence_2}, \ref{fig:compoundPoisson_convergence_3} and \ref{fig:compoundPoisson_convergence_4} respectively. Good performance can be seen for the $\lambda=1$ case with convergence of all 3 parameters within a step or two. The Cram\'er-Rao lower bound captured the spread of the estimates well.

However for $\lambda=10$ and $\lambda=100$ case, the estimates of $\alpha$ and $\beta$ struggled to converge and increased/decreased without bounds without affecting the log likelihood. It appears the EM algorithm failed for $\lambda>10$ looking at these particular examples.

\begin{figure}[p]
	\centering
    \centerline{
    \begin{subfigure}[b]{0.49\textwidth}
        \includegraphics[width=\textwidth]{../figures/compoundPoisson/convergence_1_lnL.eps}
        \caption{Log likelihood}
    \end{subfigure}
        \begin{subfigure}[b]{0.49\textwidth}
        \includegraphics[width=\textwidth]{../figures/compoundPoisson/convergence_1_lambda.eps}
        \caption{$\lambda$}
    \end{subfigure}
    }
    \centerline{
    \begin{subfigure}[b]{0.49\textwidth}
        \includegraphics[width=\textwidth]{../figures/compoundPoisson/convergence_1_alpha.eps}
        \caption{$\alpha$}
    \end{subfigure}
        \begin{subfigure}[b]{0.49\textwidth}
        \includegraphics[width=\textwidth]{../figures/compoundPoisson/convergence_1_beta.eps}
        \caption{$\beta$}
    \end{subfigure}
    }
    \caption{EM algorithm was used to estimate the parameters of a $\CPoisson(1,1,1)$ random variable using 1\,000 simulated samples, repeated 10 times. The graphs show the log likelihood and the estimated parameters at each EM step for each repeat of the experiment. The dashed lines show the standard deviation using the Cram\'er-Rao lower bound.}
    \label{fig:compoundPoisson_convergence_1}
\end{figure}

\begin{figure}[p]
	\centering
    \centerline{
    \begin{subfigure}[b]{0.49\textwidth}
        \includegraphics[width=\textwidth]{../figures/compoundPoisson/convergence_2_lnL.eps}
        \caption{Log likelihood}
    \end{subfigure}
        \begin{subfigure}[b]{0.49\textwidth}
        \includegraphics[width=\textwidth]{../figures/compoundPoisson/convergence_2_lambda.eps}
        \caption{$\lambda$}
    \end{subfigure}
    }
    \centerline{
    \begin{subfigure}[b]{0.49\textwidth}
        \includegraphics[width=\textwidth]{../figures/compoundPoisson/convergence_2_alpha.eps}
        \caption{$\alpha$}
    \end{subfigure}
        \begin{subfigure}[b]{0.49\textwidth}
        \includegraphics[width=\textwidth]{../figures/compoundPoisson/convergence_2_beta.eps}
        \caption{$\beta$}
    \end{subfigure}
    }
    \caption{EM algorithm was used to estimate the parameters of a $\CPoisson(10,1,1)$ random variable using 1\,000 simulated samples, repeated 10 times. The graphs show the log likelihood and the estimated parameters at each EM step for each repeat of the experiment. The dashed lines show the standard deviation using the Cram\'er-Rao lower bound.}
    \label{fig:compoundPoisson_convergence_2}
\end{figure}

\begin{figure}[p]
	\centering
    \centerline{
    \begin{subfigure}[b]{0.49\textwidth}
        \includegraphics[width=\textwidth]{../figures/compoundPoisson/convergence_3_lnL.eps}
        \caption{Log likelihood}
    \end{subfigure}
        \begin{subfigure}[b]{0.49\textwidth}
        \includegraphics[width=\textwidth]{../figures/compoundPoisson/convergence_3_lambda.eps}
        \caption{$\lambda$}
    \end{subfigure}
    }
    \centerline{
    \begin{subfigure}[b]{0.49\textwidth}
        \includegraphics[width=\textwidth]{../figures/compoundPoisson/convergence_3_alpha.eps}
        \caption{$\alpha$}
    \end{subfigure}
        \begin{subfigure}[b]{0.49\textwidth}
        \includegraphics[width=\textwidth]{../figures/compoundPoisson/convergence_3_beta.eps}
        \caption{$\beta$}
    \end{subfigure}
    }
    \caption{EM algorithm was used to estimate the parameters of a $\CPoisson(1,100,1)$ random variable using 1\,000 simulated samples, repeated 10 times. The graphs show the log likelihood and the estimated parameters at each EM step for each repeat of the experiment. The dashed lines show the standard deviation using the Cram\'er-Rao lower bound.}
    \label{fig:compoundPoisson_convergence_3}
\end{figure}

\begin{figure}[p]
	\centering
    \centerline{
    \begin{subfigure}[b]{0.49\textwidth}
        \includegraphics[width=\textwidth]{../figures/compoundPoisson/convergence_4_lnL.eps}
        \caption{Log likelihood}
    \end{subfigure}
        \begin{subfigure}[b]{0.49\textwidth}
        \includegraphics[width=\textwidth]{../figures/compoundPoisson/convergence_4_lambda.eps}
        \caption{$\lambda$}
    \end{subfigure}
    }
    \centerline{
    \begin{subfigure}[b]{0.49\textwidth}
        \includegraphics[width=\textwidth]{../figures/compoundPoisson/convergence_4_alpha.eps}
        \caption{$\alpha$}
    \end{subfigure}
        \begin{subfigure}[b]{0.49\textwidth}
        \includegraphics[width=\textwidth]{../figures/compoundPoisson/convergence_4_beta.eps}
        \caption{$\beta$}
    \end{subfigure}
    }
    \caption{EM algorithm was used to estimate the parameters of a $\CPoisson(100,100,1)$ random variable using 1\,000 simulated samples, repeated 10 times. The graphs show the log likelihood and the estimated parameters at each EM step for each repeat of the experiment. The dashed lines show the standard deviation using the Cram\'er-Rao lower bound.}
    \label{fig:compoundPoisson_convergence_4}
\end{figure}

\section{Failure Evaluation}
It should be convincing that the EM algorithm failed when the compound-Poisson Gamma random variable starts behaving Normally. In particular in Figure \ref{fig:compoundPoisson_convergence_4}, estimates of $\lambda$ are stable while estimates of $\alpha$ and $\beta$ struggled to converge. This could be because as the compound-Poisson Gamma random variable approach being Normal, the parameters $(\lambda,\alpha,\beta)$ becomes degenerate and there is more than one way to represent a two parameter random variable $\normal(\mu,\sigma^2)$.

One way to see this is to look at the Newton-Raphson step, in the M step, in particular the Hessian matrix $\nabla_{\alpha,\beta}\nabla_{\alpha,\beta}\T T$ for high $\lambda$. The elements of the Hessian matrix can be found in Equations \eqref{eq:compoundPoisson:d2tdadb}, \eqref{eq:compoundPoisson:d2td2b} and \eqref{eq:compoundPoisson:d2td2a}. The $\dfrac{\partial^2T}{\partial\alpha^2}$ element is 
\begin{multline*}
	\frac{\partial^2 T}{\partial \alpha^2} = 
	\sum_{i=1}^n\mathbb{I}(x_i>0)\left[
		-(y_i^2+\zeta_i)\psi'(\alpha y_i) - 2\zeta_i\alpha y_i\psi''(\alpha y_i)
		\vphantom{\frac{1}{2}}
	\right.
	\\
	\left.	
		-\frac{1}{2}\zeta_i\alpha^2y_i^2\psi'''(\alpha y_i)
	\right]
	\ .
\end{multline*}
For high $\alpha$ and high $\lambda$, and hence high $y_i$'s, an approximation can be used for the polygamma functions $\psi^{(k)}(n)$. Firstly using Stirling's approximation
\begin{equation}
	\ln\Gamma(n)\approx\ln(n!)\approx n\ln n-n
\end{equation}
then
\begin{equation}
	\psi(n) = \frac{\partial\ln\Gamma(n)}{\partial n} \approx \ln n
	\ .
\end{equation}
Differentiating further
\begin{align}
	\psi'(n) &\approx 1/n \\
	\psi''(n) & \approx -1/n^2 \\
	\psi'''(n) & \approx 2/n^3
	\ .
\end{align}
Using the approximations, $\dfrac{\partial^2T}{\partial\alpha^2}$ can be approximated
\begin{equation*}
	\frac{\partial^2 T}{\partial \alpha^2} \approx 
	\sum_{i=1}^n\mathbb{I}(x_i>0)
	\left[
		-\frac{y_i^2+\zeta_i}{\alpha y_i} + 2\frac{\zeta_i\alpha y_i}{\alpha^2 y_i^2}
		-\frac{1}{2}\frac{2\zeta_i\alpha^2y_i^2}{\alpha^3 y_i^3}
	\right]
\end{equation*}
to get
\begin{equation}
	\frac{\partial^2 T}{\partial \alpha^2} \approx 
	\sum_{i=1}^n\mathbb{I}(x_i>0)
	\left[
		-\frac{y_i}{\alpha}
	\right]
	\ .
\end{equation}

The Hessian matrix is then
\begin{equation}
	\nabla_{\alpha,\beta}\nabla_{\alpha,\beta}\T T \approx
	\sum_{i=1}^n
	\mathbb{I}(x_i>0)
	\begin{pmatrix}
		-y_i/\alpha  & y_i/\beta \\
		y_i/\beta & -\alpha y_i/\beta^2
	\end{pmatrix}
	\ . 
\end{equation}
The determinant of the Hessian matrix is then
\begin{equation*}
	\|\nabla_{\alpha,\beta}\nabla_{\alpha,\beta}\T T\|
	\approx
	\left(\sum_{i=1}^n-\frac{\alpha y_i}{\beta^2}\right)
	\left(\sum_{i=1}^n-\frac{y_i}{\alpha}\right) - \left(\sum_{i=1}^n\frac{y_i}{\beta}\right)^2
\end{equation*}
\begin{equation*}
	\|\nabla_{\alpha,\beta}\nabla_{\alpha,\beta}\T T\|
	\approx
	\left(\sum_{i=1}^n\sum_{j=1}^n\frac{y_iy_j}{\beta^2}\right)
	 - \left(\sum_{i=1}^n\frac{y_i}{\beta}\right)\left(\sum_{j=1}^n\frac{y_j}{\beta}\right)
\end{equation*}
\begin{equation*}
	\|\nabla_{\alpha,\beta}\nabla_{\alpha,\beta}\T T\|
	\approx
	\left(\sum_{i=1}^n\sum_{j=1}^n\frac{y_iy_j}{\beta^2}\right)
	 - \left(\sum_{i=1}^n\sum_{j=1}^n\frac{y_iy_j}{\beta^2}\right)
\end{equation*}
to get
\begin{equation}
	\|\nabla_{\alpha,\beta}\nabla_{\alpha,\beta}\T T\|
	\approx
	0
	\ .
\end{equation}

This results in a few things. Firstly $\nabla_{\alpha,\beta}\nabla_{\alpha,\beta}\T T$ is singular thus its inverse cannot be evaluated, needed for the Newton-Raphson method. Secondly, the sufficient conditions to classify a stationary point as a maximum are that the diagonal elements of the Hessian matrix are negative and the determinate is positive. For high $\lambda$ and $\alpha$, the second condition is not met.

Another argument which can be made is to restrict the parameter $\beta$ for a given mean to be
\begin{equation}
	\beta = \frac{\lambda\alpha}{\widehat{\mu}}
	\label{eq:compoundPoisson:beta_restrict}
\end{equation}
where
\begin{equation}
	\widehat{\mu} = \frac{1}{n}\sum_{i=1}^n x_i
\end{equation}
and investigate whenever a unique solution to $\dfrac{\partial T}{\partial \alpha} = 0$ can be found for high $\lambda$ and $\alpha$. The constrained objective function is
\begin{multline*}
	T(\lambda,\alpha)
	=
	-n\lambda+
	\sum_{i=1}^n
	\mathbb{I}(x_i>0)
	\left[
	y_i\alpha
	\ln\left(\frac{\lambda\alpha}{\widehat{\mu}}\right)-\ln\Gamma(\alpha y_i) - \frac{1}{2}\zeta_i\alpha^2\psi'(y_i\alpha)
	\right.
	\\
	\left.
		+y_i\alpha\ln x_i - \frac{\lambda\alpha x_i}{\widehat{\mu}}
		+ y_i \ln \lambda
		\vphantom{\frac{1}{1}}
	\right]
	+ c
\end{multline*}
which can be approximated to
\begin{multline*}
	T(\lambda,\alpha)=
	-n\lambda+
	\sum_{i=1}^n
	\mathbb{I}(x_i>0)
	\left[
		y_i\alpha\ln\left(\frac{\lambda\alpha}{\widehat{\mu}}\right)-\alpha y_i\ln(\alpha y_i) + \alpha y_i - \frac{1}{2}\frac{\zeta_i\alpha}{y_i}
	\right.
	\\
	\left.
		+y_i\alpha\ln x_i - \frac{\lambda\alpha x_i}{\widehat{\mu}}
		+ y_i \ln \lambda
		\vphantom{\frac{1}{1}}
	\right]
	+ c
\end{multline*}
\begin{multline*}
	T(\lambda,\alpha)=
	-n\lambda+
	\sum_{i=1}^n
	\mathbb{I}(x_i>0)
	\left[
		y_i\alpha\ln\left(\frac{\lambda}{y_i\widehat{\mu}}\right) + \alpha y_i - \frac{1}{2}\frac{\zeta_i\alpha}{y_i}
	\right.
	\\
	\left.
		+y_i\alpha\ln x_i - \frac{\lambda\alpha x_i}{\widehat{\mu}}
		+ y_i \ln \lambda
		\vphantom{\frac{1}{1}}
	\right]
	+ c
	\ .
\end{multline*}
Taking the derivative with respect to $\alpha$
\begin{equation*}
	\frac{\partial T}{\partial \alpha} =
	\sum_{i=1}^n
	\mathbb{I}(x_i>0)
	\left[
		y_i\ln\left(\frac{\lambda}{y_i\widehat{\mu}}\right)
		+ y_i - \frac{1}{2}\frac{\zeta_i}{y_i}
		+y_i\ln x_i - \frac{\lambda x_i}{\widehat{\mu}}
		\vphantom{\frac{1}{1}}
	\right]
\end{equation*}
and this is not $\alpha$ dependent, thus there is no solution to $\alpha$.

\chapter{Variance Prediction}
In the previous chapter, it was found that it was hard to fit a compound Poisson random variable onto the grey values directly itself. Results did show that the variance of the grey value has a linear relationship with the mean grey value. As a result, it would be possible to predict the variance of the grey value, given the grey value. This opens up new ways to predict the uncertainty of each individual pixel in an x-ray projection.

In this chapter, generalised linear models \citep{nelder1972generalized, nelder1972generalized_2, mccullagh1984generalized} with different link functions and polynomial explanatory variables were selected using forward step-wise selection. These selected models were compared using cross validation to find the best model.

\section{Generalised Linear Models}

The sample variance-mean data were obtained from the replicated projections. Let $x_{x,y}^{(r)}$ be the grey value of the pixel at $(x,y)$ from the $r$th replicate projection for $r=1,2,3,\dotdotdot,n$. The pixel sample mean is
\begin{equation}
    \overline{x}_{x,y}=\frac{1}{n}\sum_{r=1}^n x_{x,y}^{(r)}
\end{equation}
and the pixel sample variance is
\begin{equation}
    y_{x,y} =
    \frac{1}{n-1}
    \sum_{r=1}^n
        \left(
            x_{x,y}^{(r)} - \overline{x}_{x,y}
        \right)^2
    \ .
\end{equation}
Only pixels in the region of interest (ROI) were considered here, that is pixels which represent the test sample.

The aim is to model and predict the pixel variance given the pixel grey value by fitting a model onto the sample variance-mean data. No spatial information is used, thus the variance-mean data is denoted as $(x_1,y_1),(x_2,y_2),\cdots,(x_m,y_m)$ from now on, where $m$ is the area of the ROI or the size of the dataset. Let $Y(x)$ be a random variable and the sample variance given a grey value mean $x$. It was assumed the standard error from estimating the mean was negligible so that the uncertainty is captured by the random variable $Y$. It was assumed that for a given pixel, the grey values are Normal and i.i.d. Let $\sigma^2(x)$ be the variance given a grey value $x$, then it can be shown that
\begin{equation}
\dfrac{(n-1)Y(x)}{\sigma^2(x)}\sim\chi^2_{n-1}
\end{equation}
which results in
\begin{equation}
Y(x)\sim\gammaDist\left(\alpha,\dfrac{\alpha}{\sigma^2(x)}\right)
\end{equation}
where $\alpha=(n-1)/2$ is the shape parameter. The expectation and variance are
\begin{equation}
\expectation\left[Y(x)\right]=\sigma^2(x)
\end{equation}
and
\begin{equation}
\variance\left[Y(x)\right]=\dfrac{\sigma^2(x)}{\alpha}
\end{equation}
respectively.

This framework allows the use of generalised linear models (GLM) \citep{nelder1972generalized,nelder1972generalized_2, mccullagh1984generalized}. In the gamma distribution case, GLM can be used to model
\begin{equation}
Y(x)\sim\gammaDist\left(\alpha,\dfrac{\alpha}{g^{-1}(\eta(x))}\right)
\end{equation}
where $g(y)$ is the link function and $\eta(x)$ is a linear function called the systematic component. It should be noted that
\begin{equation}
  \expectation\left[Y(x)\right]=g^{-1}(\eta(x))
\end{equation}
which shows how the link function and systematic component work together. Examples of link functions are the identity link
\begin{equation}
g(y)=g^{-1}(y)=y
\end{equation}
and, for the gamma distribution case, the canonical link
\begin{equation}
g(y)=g^{-1}(y)=1/y \ .
\end{equation}
An example of a systematic component are polynomials $\eta(x)=\beta_0+\sum_{j=1}^{p-1}\beta_j x^j$ so that when used with the identity link, for example, $\expectation\left[Y(x)\right] = \beta_0+\sum_{j=1}^{p-1}\beta_j x^j$. Iterative reweighted least squares \citep{friedman2001elements} can be used to estimate the parameters $\beta_0, \beta_1, \cdots, \beta_{p-1}$ given data for the model to fit onto.

Once the parameters have been estimated, prediction of the variance given a grey value $x$ is done using the expectation $\widehat{y}(x) = \expectation\left[Y(x)\right] = g^{-1}(\eta(x))$.

\section{Model Selection}

This section describes how forward stepwise selection \citep{friedman2001elements} was used to select which polynomials to use in the systematic component.

In summary, forward stepwise selection fits a basic model to the data initially. A term is added to make the model more complicated at each step to improve the fit onto the data. This is continued until the model cannot be improved subject to overfitting. The Akaike information criterion (AIC) \citep{friedman2001elements} and the Bayesian information criterion (BIC) \citep{friedman2001elements} are criteria which can be used to assess the fit of the model at each step without overfitting to the data too much.

The AIC and BIC are given as
\begin{equation}
\AIC = 2p-2\ln L
\end{equation}
and
\begin{equation}
\BIC = p\ln m - 2\ln L
\end{equation}
respectively where $p$ is the number of terms in the systematic component and $\ln L$ is the log likelihood of the GLM. The model with the lowest AIC or BIC is preferred. GLM aims to maximise the log likelihood but the additional terms in the criteria penalise models with too many terms. The log likelihood is given as
\begin{equation}
  \ln L = \sum_{i=1}^m \left[
    \alpha\ln\alpha
    -\ln\Gamma(\alpha)
    -\alpha\ln \widehat{y}_i
    +(\alpha-1)\ln y_i
    -\frac{\alpha y_i}{\widehat{y}_i}
  \right]
  \ .
\end{equation}
where $\widehat{y}_i=\widehat{y}(x_i)$. It should be reminded that $\alpha=(n-1)/2$ was assumed to be known and does not need to be estimated.

The procedure is as follows. A criterion and a link function is chosen beforehand. In the initial step, a GLM with systematic component $\eta(x)=\beta_0$ is fitted and the criterion is recorded. In the next step, a polynomial term with one order higher is added to the systematic component $\eta(x)=\beta_0+\beta_1 x$, fitted and the criterion recorded. In addition, a polynomial term with one order lower is added $\beta(x)=\beta_{-1}x^{-1}+\beta_0$ and fitted separately with the criterion recorded. The model which decreases the criterion the most is accepted. Adding higher and lower order polynomials to the systematic component is repeated, for example after accepting $\eta(x)=\beta_0+\beta_1 x$, the following systematic components $\eta(x)=\beta_0+\beta_1 x+\beta_2x^2$ and $\eta(x)=\beta_{-1}x^{-1}+\beta_0+\beta_1 x$ are fitted and assessed. This is repeated until the criterion cannot be decreased and the procedure is left with the final model.

Forward stepwise selection was conducted on the datasets \texttt{AbsNoFilterDeg120} and \texttt{AbsFilterDeg120}. The procedure was repeated 100 times by using a random permutation with replacement of the replicated projections to obtain a different sample variance-mean data which introduces some variation to the data. The procedure was also repeated using various shading corrections to investigate the effects of shading correction on the variance-mean relationship.

\begin{sidewaystable}
\centering
\begin{tabular}{cc|ccc}
\multicolumn{2}{c|}{Identity Link}& null & bw & linear \\ \hline
\multirow{3}{*}{AIC} & order     & \inputNumber{../figures/varmean/GlmSelectAicAbsNoFilterNull_identityorder.txt}     & \inputNumber{../figures/varmean/GlmSelectAicAbsNoFilterBw_identityorder.txt}     & \inputNumber{../figures/varmean/GlmSelectAicAbsNoFilterLinear_identityorder.txt}     \\
                     & votes     & \inputNumber{../figures/varmean/GlmSelectAicAbsNoFilterNull_identityvote.txt}      & \inputNumber{../figures/varmean/GlmSelectAicAbsNoFilterBw_identityvote.txt}      & \inputNumber{../figures/varmean/GlmSelectAicAbsNoFilterLinear_identityvote.txt}      \\
                     & criterion & \inputNumber{../figures/varmean/GlmSelectAicAbsNoFilterNull_identitycriterion.txt} & \inputNumber{../figures/varmean/GlmSelectAicAbsNoFilterBw_identitycriterion.txt} & \inputNumber{../figures/varmean/GlmSelectAicAbsNoFilterLinear_identitycriterion.txt} \\ \hline
\multirow{3}{*}{BIC} & order     & \inputNumber{../figures/varmean/GlmSelectBicAbsNoFilterNull_identityorder.txt}     & \inputNumber{../figures/varmean/GlmSelectBicAbsNoFilterBw_identityorder.txt}     & \inputNumber{../figures/varmean/GlmSelectBicAbsNoFilterLinear_identityorder.txt}     \\
                     & votes     & \inputNumber{../figures/varmean/GlmSelectBicAbsNoFilterNull_identityvote.txt}      & \inputNumber{../figures/varmean/GlmSelectBicAbsNoFilterBw_identityvote.txt}      & \inputNumber{../figures/varmean/GlmSelectBicAbsNoFilterLinear_identityvote.txt}      \\
                     & criterion & \inputNumber{../figures/varmean/GlmSelectBicAbsNoFilterNull_identitycriterion.txt} & \inputNumber{../figures/varmean/GlmSelectBicAbsNoFilterBw_identitycriterion.txt} & \inputNumber{../figures/varmean/GlmSelectBicAbsNoFilterLinear_identitycriterion.txt} \\
\\
\multicolumn{2}{c|}{Canonical Link}& null & bw & linear \\ \hline
\multirow{3}{*}{AIC} & order      & \inputNumber{../figures/varmean/GlmSelectAicAbsNoFilterNull_reciprocalorder.txt}     & \inputNumber{../figures/varmean/GlmSelectAicAbsNoFilterBw_reciprocalorder.txt}     & \inputNumber{../figures/varmean/GlmSelectAicAbsNoFilterLinear_reciprocalorder.txt}     \\
                     & votes      & \inputNumber{../figures/varmean/GlmSelectAicAbsNoFilterNull_reciprocalvote.txt}      & \inputNumber{../figures/varmean/GlmSelectAicAbsNoFilterBw_reciprocalvote.txt}      & \inputNumber{../figures/varmean/GlmSelectAicAbsNoFilterLinear_reciprocalvote.txt}      \\
                     & criterion  & \inputNumber{../figures/varmean/GlmSelectAicAbsNoFilterNull_reciprocalcriterion.txt} & \inputNumber{../figures/varmean/GlmSelectAicAbsNoFilterBw_reciprocalcriterion.txt} & \inputNumber{../figures/varmean/GlmSelectAicAbsNoFilterLinear_reciprocalcriterion.txt} \\ \hline
\multirow{3}{*}{BIC} & order      & \inputNumber{../figures/varmean/GlmSelectBicAbsNoFilterNull_reciprocalorder.txt}     & \inputNumber{../figures/varmean/GlmSelectBicAbsNoFilterBw_reciprocalorder.txt}     & \inputNumber{../figures/varmean/GlmSelectBicAbsNoFilterLinear_reciprocalorder.txt}     \\
                     & votes      & \inputNumber{../figures/varmean/GlmSelectBicAbsNoFilterNull_reciprocalvote.txt}      & \inputNumber{../figures/varmean/GlmSelectBicAbsNoFilterBw_reciprocalvote.txt}      & \inputNumber{../figures/varmean/GlmSelectBicAbsNoFilterLinear_reciprocalvote.txt}      \\
                     & criterion  & \inputNumber{../figures/varmean/GlmSelectBicAbsNoFilterNull_reciprocalcriterion.txt} & \inputNumber{../figures/varmean/GlmSelectBicAbsNoFilterBw_reciprocalcriterion.txt} & \inputNumber{../figures/varmean/GlmSelectBicAbsNoFilterLinear_reciprocalcriterion.txt}     
\end{tabular}
\caption{Forward stepwise selection was used to find suitable polynomial orders when fitting a GLM onto the sample variance-mean data of the grey values from the projections in \texttt{AbsNoFilterDeg120}. The columns of the table represent different shading corrections used on the projections. Forward stepwise selection was repeated 100 times by using a different random permutation with replacement to obtain a different sample variance-mean data. The row labelled order shows the most common selected polynomial orders. The error bars are the standard deviation from the 100 repeats.}
\label{table:meanVar_glmselect_absnofilter}
\end{sidewaystable}

\begin{sidewaystable}
\centering
\begin{tabular}{cc|ccc}
\multicolumn{2}{c|}{Identity Link}& null & bw & linear \\ \hline
\multirow{3}{*}{AIC} & order     & \inputNumber{../figures/varmean/GlmSelectAicAbsFilterNull_identityorder.txt}     & \inputNumber{../figures/varmean/GlmSelectAicAbsFilterBw_identityorder.txt}     & \inputNumber{../figures/varmean/GlmSelectAicAbsFilterLinear_identityorder.txt}     \\
                     & votes     & \inputNumber{../figures/varmean/GlmSelectAicAbsFilterNull_identityvote.txt}      & \inputNumber{../figures/varmean/GlmSelectAicAbsFilterBw_identityvote.txt}      & \inputNumber{../figures/varmean/GlmSelectAicAbsFilterLinear_identityvote.txt}      \\
                     & criterion & \inputNumber{../figures/varmean/GlmSelectAicAbsFilterNull_identitycriterion.txt} & \inputNumber{../figures/varmean/GlmSelectAicAbsFilterBw_identitycriterion.txt} & \inputNumber{../figures/varmean/GlmSelectAicAbsFilterLinear_identitycriterion.txt} \\ \hline
\multirow{3}{*}{BIC} & order     & \inputNumber{../figures/varmean/GlmSelectBicAbsFilterNull_identityorder.txt}     & \inputNumber{../figures/varmean/GlmSelectBicAbsFilterBw_identityorder.txt}     & \inputNumber{../figures/varmean/GlmSelectBicAbsFilterLinear_identityorder.txt}     \\
                     & votes     & \inputNumber{../figures/varmean/GlmSelectBicAbsFilterNull_identityvote.txt}      & \inputNumber{../figures/varmean/GlmSelectBicAbsFilterBw_identityvote.txt}      & \inputNumber{../figures/varmean/GlmSelectBicAbsFilterLinear_identityvote.txt}      \\
                     & criterion & \inputNumber{../figures/varmean/GlmSelectBicAbsFilterNull_identitycriterion.txt} & \inputNumber{../figures/varmean/GlmSelectBicAbsFilterBw_identitycriterion.txt} & \inputNumber{../figures/varmean/GlmSelectBicAbsFilterLinear_identitycriterion.txt} \\
\\
\multicolumn{2}{c|}{Canonical Link}& null & bw & linear \\ \hline
\multirow{3}{*}{AIC} & order      & \inputNumber{../figures/varmean/GlmSelectAicAbsFilterNull_reciprocalorder.txt}     & \inputNumber{../figures/varmean/GlmSelectAicAbsFilterBw_reciprocalorder.txt}     & \inputNumber{../figures/varmean/GlmSelectAicAbsFilterLinear_reciprocalorder.txt}     \\
                     & votes      & \inputNumber{../figures/varmean/GlmSelectAicAbsFilterNull_reciprocalvote.txt}      & \inputNumber{../figures/varmean/GlmSelectAicAbsFilterBw_reciprocalvote.txt}      & \inputNumber{../figures/varmean/GlmSelectAicAbsFilterLinear_reciprocalvote.txt}      \\
                     & criterion  & \inputNumber{../figures/varmean/GlmSelectAicAbsFilterNull_reciprocalcriterion.txt} & \inputNumber{../figures/varmean/GlmSelectAicAbsFilterBw_reciprocalcriterion.txt} & \inputNumber{../figures/varmean/GlmSelectAicAbsFilterLinear_reciprocalcriterion.txt} \\ \hline
\multirow{3}{*}{BIC} & order      & \inputNumber{../figures/varmean/GlmSelectBicAbsFilterNull_reciprocalorder.txt}     & \inputNumber{../figures/varmean/GlmSelectBicAbsFilterBw_reciprocalorder.txt}     & \inputNumber{../figures/varmean/GlmSelectBicAbsFilterLinear_reciprocalorder.txt}     \\
                     & votes      & \inputNumber{../figures/varmean/GlmSelectBicAbsFilterNull_reciprocalvote.txt}      & \inputNumber{../figures/varmean/GlmSelectBicAbsFilterBw_reciprocalvote.txt}      & \inputNumber{../figures/varmean/GlmSelectBicAbsFilterLinear_reciprocalvote.txt}      \\
                     & criterion  & \inputNumber{../figures/varmean/GlmSelectBicAbsFilterNull_reciprocalcriterion.txt} & \inputNumber{../figures/varmean/GlmSelectBicAbsFilterBw_reciprocalcriterion.txt} & \inputNumber{../figures/varmean/GlmSelectBicAbsFilterLinear_reciprocalcriterion.txt}     
\end{tabular}
\caption{Forward stepwise selection was used to find suitable polynomial orders when fitting a GLM onto the sample variance-mean data of the grey values from the projections in \texttt{AbsFilterDeg120}. The columns of the table represent different shading corrections used on the projections. Forward stepwise selection was repeated 100 times by using a different random permutation with replacement to obtain a different sample variance-mean data. The row labelled order shows the most common selected polynomial orders. The error bars are the standard deviation from the 100 repeats.}
\label{table:meanVar_glmselect_absfilter}
\end{sidewaystable}

The results are shown in Tables \ref{table:meanVar_glmselect_absnofilter} and \ref{table:meanVar_glmselect_absfilter}. The models selected all have only two terms and are quite simple. Different shading correction or different criteria had no effect on the selected model. The method is quite robust to the variation introduced to the dataset when repeating the experiment because all 100 repeats consistently selected the same model.

There was some variation to the selected models between datasets. For example when using the identity link, \texttt{AbsNoFilterDeg120} prefers $\eta(x)=\beta_0+\beta_1 x$ whereas \texttt{AbsFilterDeg120} prefers $\eta(x)=\beta_{-1}x^{-1}+\beta_0$. Using the canonical link, both datasets selected $\eta(x)=\beta_{-1}x^{-1}+\beta_0$ which correspond to $\widehat{y}(x)=\left(\beta_{-1}x^{-1}+\beta_0\right)^{-1}$. Figures \ref{fig:meanVar_varMeanExample_AbsNoFilterDeg120} and \ref{fig:meanVar_varMeanExample_AbsFilterDeg120} shows the GLM fits. There is a leverage towards the low grey values because there are more low grey values in the dataset. As a result, the GLM does not capture the curvature towards the high grey values.

\begin{figure}
  \centering
    \centerline{
    \begin{subfigure}[b]{0.49\textwidth}
        \includegraphics[width=\textwidth]{../figures/varmean/varMeanExample_AbsNoFilterDeg120_identity000010.eps}
        \caption{$\widehat{y}(x)=\beta_0+\beta_1 x$}
    \end{subfigure}
    \begin{subfigure}[b]{0.49\textwidth}
        \includegraphics[width=\textwidth]{../figures/varmean/varMeanExample_AbsNoFilterDeg120_reciprocal000100.eps}
        \caption{$\widehat{y}(x)=(\beta_{-1}x^{-1}\beta_0)^{-1}$}
    \end{subfigure}
    }
    \caption{Log frequency density histogram of the sample variance-mean data from \texttt{AbsNoFilterDeg120} with linear shading correction. The red solid line shows the GLM fit along with $\Phi(\pm 1)$ quantile error bars as dashed lines.}
    \label{fig:meanVar_varMeanExample_AbsNoFilterDeg120}
\end{figure}

\begin{figure}
  \centering
    \centerline{
    \begin{subfigure}[b]{0.49\textwidth}
        \includegraphics[width=\textwidth]{../figures/varmean/varMeanExample_AbsFilterDeg120_identity000100.eps}
        \caption{$\widehat{y}(x)=\beta_{-1}x^{-1}+\beta_0$}
    \end{subfigure}
    \begin{subfigure}[b]{0.49\textwidth}
        \includegraphics[width=\textwidth]{../figures/varmean/varMeanExample_AbsFilterDeg120_reciprocal000100.eps}
        \caption{$\widehat{y}(x)=(\beta_{-1}x^{-1}+\beta_0)^{-1}$}
    \end{subfigure}
    }
    \caption{Log frequency density histogram of the sample variance-mean data from \texttt{AbsFilterDeg120} with linear shading correction. The red solid line shows the GLM fit along with $\Phi(\pm 1)$ quantile error bars as dashed lines.}
    \label{fig:meanVar_varMeanExample_AbsFilterDeg120}
\end{figure}

\section{Cross Validation}

Instead of assessing the model fit using the AIC and BIC, the performance on predicting the variance given a grey value can be assessed using cross validation \citep{friedman2001elements}. Cross validation assess the model to make predictions on data it has not seen before. This is done by randomly splitting the $m$ data points into two disjoint sets, the training set and the test set. It was chosen that the training and test set are of the same size. The model is fitted onto the training set. Afterwards, the fitted model predict the variances given the grey values in the test set which are compared to the actual variances. 

The mean scaled deviance was used to assess the performance of variance prediction. The deviance is defined to be
\begin{equation}
  D = 2\left(
    \ln L_\mathrm{s} - \ln L    
  \right)
\end{equation}
where $L$ and $L_\mathrm{s}$ are the likelihood and saturated likelihood respectively.  The saturated likelihood is obtained by replacing all $\widehat{y}_i$ with $y_i$ in the likelihood so that
\begin{equation}
  \ln L_\mathrm{s} = \sum_{i=1}^m \left[
    \alpha\ln\alpha
    -\ln\Gamma(\alpha)
    -\alpha\ln{y_i}
    +(\alpha-1)\ln y_i
    -\alpha
  \right]
  \ .
\end{equation}
Following from this, the deviance is
\begin{equation}
  D = 2\alpha
  \sum_{i=1}^m\left[
      \dfrac{
          y_i-\widehat{y}_i
      }
      {
          \widehat{y}_i
      }
      - \ln\left(\dfrac{y_i}{\widehat{y}_i}\right)
  \right]
  \ .
\end{equation}
The mean scaled deviance is obtained by removing the factor of $\alpha$ and dividing by $m$ to get
\begin{equation}
    D_\mathrm{s} = \dfrac{2}{m}
    \sum_{i=1}^m\left[
        \dfrac{
            y_i-\widehat{y}_i
        }
        {
            \widehat{y}_i
        }
        - \ln\left(\dfrac{y_i}{\widehat{y}_i}\right)
    \right]
    \ .
\end{equation}

\begin{figure}
  \centering
  \includegraphics[width=0.49\textwidth]{../figures/varmean/devianceGraph.eps}
  \caption{Scaled deviance loss function}
  \label{fig:meanVar_deviance}
\end{figure}

The mean scaled deviance is a loss function which increases as $y_i/\widehat{y}_i$ deviates from one, this is shown in Figure \ref{fig:meanVar_deviance} and it should be noted the $x$-axis is in log scale. Another way to show this is by letting $r_i = y_i/\widehat{y}_i$ and $d_i = 2\left[r-1-\ln r\right]$ be an element from the sum in the deviance. For $r_i\approx 1$, $\ln(r)\approx(r-1)-(r-1)^2/2$ which implies $d_i\approx (r-1)^2$ with minimum at one. For $r_i$ deviate greatly from one, the loss function is asymmetric. For example a ratio of $r_i=10^1$ has a greater penalty than $r_i=10^{-1}$. This means in extreme cases, overestimates are penalised less than underestimates relative to $y_i$. 

Assuming the model is correct, it is given that $D\sim\chi_{m-p}^2$ which implies that for large $m$,
\begin{equation}
\expectation\left[D_\mathrm{s}\right] = \dfrac{1}{\alpha}
\end{equation}
and
\begin{equation}
\variance\left[D_\mathrm{s}\right] = \dfrac{2}{\alpha^2 m} \ .
\end{equation}
Because $\alpha=(n-1)/2$, this shows that the number of replicated projections used to obtain the sample variance-mean data has an influence on the mean scaled deviance. This result can be used to estimate $\alpha$ if it is unknown.

Cross validation was performed on the datasets \texttt{AbsNoFilterDeg120} and \texttt{AbsFilterDeg120} with various different shading corrections. The models selected from forward step-wise selection in the previous section were assessed. They are $\widehat{y}(x)=\beta_0+\beta_1 x$ and $\widehat{y}(x)=\beta_{-1}x^{-1}+\beta_0$ using the identity link and $\widehat{y}(x)=(\beta_{-1}x^{-1}+\beta_0)^{-1}$ using the canonical link. The analysis was repeated 100 times by using a random permutation with replacement of the replicated projections to obtain a different sample variance-mean data which introduces some variation to the data.

\begin{figure}
  \centering
  \centerline{
  \begin{subfigure}[b]{0.49\textwidth}
      \includegraphics[width=\textwidth]{../figures/varmean/varMeanCv_AbsNoFilterDeg120_testmeanscaleddeviance.eps}
      \caption{\texttt{AbsNoFilterDeg120}}
  \end{subfigure}
  \begin{subfigure}[b]{0.49\textwidth}
      \includegraphics[width=\textwidth]{../figures/varmean/varMeanCv_AbsFilterDeg120_testmeanscaleddeviance.eps}
      \caption{\texttt{AbsFilterDeg120}}
  \end{subfigure}
  }
  \caption{Test mean scaled deviance from predicting variances in a test set using a GLM fitted onto a training set. The box plot represent the 100 repeats of the analysis by by using a random permutation with replacement of the replicated projections to work out a different sample variance-mean data.}
  \label{fig:meanVar_varMeanCv}
\end{figure}

The results from the cross validation is shown in Figure \ref{fig:meanVar_varMeanCv}. The performance of the three candidate models were very similar. One exception is the model $\widehat{y}(x)=\beta_{-1}x^{-1}+\beta_0$ fitted onto the \texttt{AbsNoFilterDeg120} dataset, the mean scaled deviance was significantly larger and this is expected from a model not favoured in the forward step-wise selection in the previous section.

Shading correction, again, did not had an effect in the analysis.

From these results, it is recommended that the relationship $\widehat{y}(x)=\beta_0+\beta_1 x$ should be used for its simple form, similar performance to other candidate models and connections to the compound Poisson.

\section{Conclusion}

It was found that a gamma GLM with relationship $\widehat{y}(x)=\beta_0+\beta_1 x$ using the identity link was a good model. The simple linear relationship is attractive because it is justified by the compound Poisson model of x-ray photon behaviour. This model is also preferred by those who follows Occam’s razor.

Non-parametric and machine learning methods may be used in the prediction of variance for more flexible models. These models are flexible, however, they are slow and are unnecessary in a low dimensional problem with a large number of data points.

\chapter{Inference}
The aim of this project is to obtain an x-ray image of a 3D printed sample and compare it with a simulation of that scan using software called \texttt{aRTist}. \texttt{aRTist} can simulate x-ray scans of the 3D printed sample given the specifications of the scan, such as the x-ray source, the x-ray detector and the blueprint of the 3D printed sample. Users of \texttt{aRTist} can align the simulated x-ray image to the real x-ray image using numerical methods, however this is outside the scope of this thesis.

Disagreement between the scan and \texttt{aRTist} can be found by simply subtracting one image from the other, any values too big in magnitude can be considered as a defect. However in the previous chapters, it was found that x-ray photons behave randomly and large differences in the comparison can be due to chance. Thus the comparisons should be done under the face of uncertainty.

A pixel by pixel inference was proposed to do defect detection. A statistic $z_{i,j}$ for the $(i,j)$ positioned pixel can be calculated for all pixels. This statistic is
\begin{equation}
    z_{i,j} = 
    \dfrac{
        \text{scan}_{i,j} - \text{aRTist}_{i,j}
    }
    {
        \sqrt{\widehat{\variance}\left[\text{scan}_{i,j}\right]}
    }
\end{equation}
where $\widehat{\variance}\left[\text{scan}_{i,j}\right]$ is the estimated grey value variance of pixel $(i,j)$ in the scan. Under mild assumptions, it was shown in the previous chapters that the grey values in the scan are Normal. Thus by treating the simulated image from \texttt{aRTist} as known, then
\begin{equation}
z_{i,j}\sim \normal(0,1) \ .
\end{equation}
This means the randomness of the $z_{i,j}$ statistics can be quantified.

The estimation of the grey value variance, that is $\widehat{\variance}\left[\text{scan}_{i,j}\right]$ will be explained here. The variance model can be calibrated, or trained, by holding out a number replicated x-ray scans of a 3D printed sample. These replicated x-ray scans provide variance-mean data which then can be used to train the variance model, such as a Gamma distributed GLM. The method on doing so was described in the previous chapter and it was found a linear relationship between the variance and the mean was a good model. The variance was then predicted using the grey value in the \texttt{aRTist} simulation.

This method for inference can be applied to the dataset \texttt{Sep16 120deg}. The 20 x-ray images were spilt into 2. 19 images were used to train the variance-mean model. One image, called the test image, was used to compare with \texttt{aRTist}, as shown in Figure \ref{fig:inference_scan_aRTist}.

\begin{figure}
	\centering
    \centerline{
    \begin{subfigure}[b]{0.49\textwidth}
        \includegraphics[width=\textwidth]{../figures/inference/scan.eps}
        \caption{X-ray image}
    \end{subfigure}
    \begin{subfigure}[b]{0.49\textwidth}
        \includegraphics[width=\textwidth]{../figures/inference/aRTist.eps}
        \caption{\texttt{aRTist} simulation}
    \end{subfigure}
    }
    \caption{An x-ray scan of a 3D printed cuboid, from the \texttt{Sep16 120deg} dataset. This can be compared directly to the \texttt{aRTist} simulation for defects.}
    \label{fig:inference_scan_aRTist}
\end{figure}

For each pixel, a $z_{i,j}$ statistic was calculated. A $z_{i,j}$ statistic too large in magnitude can be considered to be evidence of a positive result. Another way to represent the $z_{i,j}$ statistic is the $p$-value which is given as
\begin{equation}
    p_{i,j} = 2(1-\Phi(\|z_{i,j}\|))
\end{equation}
which can takes values $0\leqslant p_{i,j} \leqslant 1$. A $p$-value too small is considered to be evidence of a positive result. The result $z$ statistics and $p$-values are shown in Figure \ref{fig:logp_z}.

\begin{figure}
	\centering
    \centerline{
    \begin{subfigure}[b]{0.49\textwidth}
        \includegraphics[width=\textwidth]{../figures/inference/logp.eps}
        \caption{$-\log p\text{-values}$}
    \end{subfigure}
    \begin{subfigure}[b]{0.49\textwidth}
        \includegraphics[width=\textwidth]{../figures/inference/z_image.eps}
        \caption{$z$ statistics}
    \end{subfigure}
    }
    \caption{The resulting z statistics and $p$ values comparing an x-ray scan with the \texttt{aRTist} simulation.}
    \label{fig:logp_z}
\end{figure}

The resulting $z$ statistics and $p$-values are concerning. This is because the $p$-values are not very smooth on the surfaces of the sample. It should be expected that small $p$-values are in areas of the defects. Significant pixels were chosen for $\|z\|>\input{../figures/inference/z_critical.txt}$, this value was chosen by using the \cite{benjamini1995controlling} algorithm at the $z_\alpha = 2$ significance level. The significant pixels are shown in Figure \ref{fig:sig_pixels}.

This proposed method for defect detection failed because too many false positives were detected. These false positives appear to have some structure, for example clustering in the corners or on surfaces. In addition, false negatives were detected because not all of the defects were detected.

Model misspecification appears to be the main source of error. The $z$ statistics can be inspected using a histogram, as shown in Figure \ref{fig:z_histo}. The histogram of the $z$ statistics do not look Normal which seems to suggest that the assumption of $z_{i,j}\sim \normal(0,1)$ is incorrect. However this assumption can relaxed and can be done using the empirical null \citep{efron2004large}.

\begin{figure}
    \centering
    \includegraphics[width=0.7\textwidth]{../figures/inference/sig_pixels.eps}
    \caption{Signficiant pixels highlighted at the $z_\alpha = 2$ significance level.}
    \label{fig:sig_pixels}
\end{figure}

\begin{figure}
    \centering
    \includegraphics[width=0.7\textwidth]{../figures/inference/z_histo.eps}
    \caption{Histogram of the $z$ statistics}
    \label{fig:z_histo}
\end{figure}

This chapter will cover hypothesis testing, for a single test and then for multiple tests. The empirical null will then be reviewed which then can be extended to an image filter, called the empirical null filter. This filter will adjust each $z$ statistic according to its neighbours, ironing out false positive results. Simulations and actual results will be shown at the end of the chapter.

\section{Hypothesis Testing}

Hypothesis testing dates back to \cite{pearson1900on}, \cite{neyman1933on} and \cite{fisher1970statistical}. It is so fundamental in science that it is now taught in schools.

The focus on this section is on the single hypothesis test. Here the random variable $Z\sim\normal(\mu,1)$ will be studied here as this is of interest in this project. It also appears in other tests such as the difference in sample means.

In the context of this project, an example of hypothesis testing will be given here. Here, it was assumed that $Z\sim\normal(0,1)$. A null hypothesis can be written down to describe this
\begin{equation}
    H_0:\mu=0 \ .
\end{equation}
It specifies the assumptions made on the random variable $Z$. Any data behaving as assumed is treated as a negative result.

A positive result is obtained when unlikely data is obtained and this happens if $Z$ deviates too much from 0. This can be described as testing $H_0$ against an alternative hypothesis $H_1$ where
\begin{equation}
    H_1:\mu\neq0 \ .
\end{equation}
How much $Z$ deviates from 0 to be considered a positive is up to the user but typically $\|Z\|>z_\alpha$ where $z_\alpha =2$ is considered significant.

A way to quantify the threshold for a positive result is to use something called the size of the test, otherwise known as the significance level, denoted as $\alpha$. This is the probability of a false positive result, which can be denoted as
\begin{equation}
    \prob(\|Z\|>z_\alpha|H_0) = \alpha \ .
\end{equation}
Using the fact the Normal distribution is symmetric then
\begin{equation}
    2(1 - \Phi(z_\alpha)) = \alpha \ .
    \label{eq:inference_single_alpha}
\end{equation}
The choice of $z_\alpha=2$ will set the size of the test to be $\alpha\approx 4.55\%$. This sort of choice of threshold and size is sensible.

The $p$-value is a way to represent an observation of the data $Z=z$. Similar to the size, the $p$-value is
\begin{equation}
    p=2(1-\Phi(\|z\|)) \ .
\end{equation}
As a result, the $p$-value can be compared directly to the size of the test. That is there is a positive result if $p<\alpha$, otherwise it is a negative result.

The power of the test is defined as the probability of a true positive result. For unknown $\mu$, it is useful to investigate the power for a range of $\mu$.

\section{Multiple Hypothesis Testing}

\section{Empirical Null}

\section{Empirical Null Filter}

\section{Results}


\bibliographystyle{apalike}
\bibliography{../bib}

\end{document}
